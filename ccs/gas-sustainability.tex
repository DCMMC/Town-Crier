\section{Gas Sustainability}
\label{sec:gas-sustainability}

%Recall from Section~\ref{sec:contracts-and-gas} that computation in Ethereum is not free; it employs \emph{gas} to fuel contracts.
%For a call to a contract, that cost is paid by the user who initiates the transaction, regardless of what other contracts are called as part of execution.
As described above, Ethereum's fee model requires that gas be paid by the user who initiates a transaction, regardless of what other contracts are called as part of execution.
This means that a service which initiates calls to Ethereum contracts must spend money to execute those calls.
Without careful design, such services run the risk of malicious users (or protocol bugs) draining financial resources by provoking blockchain calls for which the service's fees will not be reimbursed.
In turn, this would lead to the service being unable to serve legitimate requests.
It is therefore critical for liveness of Ethereum-based services that they always be reimbursed for blockchain computation they initiate.

We note that the principle we define here is not Ethereum specific.
Any blockchain-based smart contract system must require fees to reimburse miners for performing and verifying computation.
If, like in Ethereum, all fees are paid by the user who initiates the computation, then this concern still applies.
To ensure that a service is not vulnerable to such attacks, we define \emph{gas sustainability}, a new condition necessary for liveness of blockchain contract-based services.

Let \bal{\wallet} denote the balance of an Ethereum wallet \wallet.

\begin{definition}[$K$-Gas Sustainability]
  \label{def:gas-sustainability}
  A service with wallet \wallet and blockchain functions $f_1, \dotsc, f_n$ is \emph{$K$-gas sustainable} if the following holds.
  When the service behaves honestly and $\bal{\wallet} \geq K$ before any $f_i$s are executed,
  then after each execution of an $f_i$ initiated by \wallet, $\bal{\wallet} \geq K$.
\end{definition}

Recall that Ethereum requires that a user have enough Ether to pay for gas expenditures.
If insufficient gas is provided, while the transaction aborts, the gas is still spent.
While Ethereum does inherently guarantee 0-gas sustainability, if a transaction is submitted by a wallet with insufficient funds, the wallet's balance will reduce to 0.
Therefore, to be $K$-gas sustainable for $K > 0$, each blockchain call made by the service must reimburse gas expenditures.
Moreover, the service must be able to afford the gas for each call since otherwise such reimbursement will be reverted with the rest of the transaction.

We show in Section~\ref{sec:protocol} how we design \tc to be gas sustainable and prove that fact in Section~\ref{sec:analysis}.

