\section{Related Work}
\label{sec:related}

Several data feeds are deployed today for smart contract systems such as Ethereum. Examples include PriceFeed~\cite{PriceFeed:2016} and Oraclize.it~\cite{Oraclize:2016}. The latter achieves distributed trust by using a second service called TLSnotary~\cite{TLSnotary}, which digitally signs TLS session data. As a result, however, unlike TC, which can flexibly tailor datagrams, Oraclize.it must serve data verbatim from a web session or API call; verbose sources thus mean superfluous data and inflated gas costs. Additionally, these services  ultimately rely on the reputations of their (small) providers to ensure data authenticity and cannot support private or custom datagrams. Alternative systems such as SchellingCoin~\cite{schellingcoin} and Augur~\cite{augur} rely on prediction markets to decentralize trust, creating a heavy reliance on human input and thus severely constraining their scope and data types.  


Despite an active developer community, research results on smart contracts are limited. Work includes off-chain contract execution for confidentiality~\cite{hawk}, and, more tangentially, exploration of e.g., randomness sources in~\cite{bonneau2015bitcoin}. The only research involving data feeds to date explores criminal applications~\cite{gyges}.

SGX is similarly in its infancy.
While a Windows SDK~\cite{sgxsdk} and programming manual~\cite{sgxmanual} have just been released, a number of pre-release papers have already explored SGX, e.g., \cite{VC3,7163052,anati2013innovative,McKeen:2013jv,Phegade:2013km}. Researchers have demonstrated applications including enclave execution of legacy (non-SGX) code~\cite{haven} and use of SGX in a distributed setting for map-reduce computations~\cite{VC3}. Several works have exposed shortcomings of the security model for SGX~\cite{sgxexplained,sgxsok,shihardwaretalk}, including side-channel attacks against enclave state. 


