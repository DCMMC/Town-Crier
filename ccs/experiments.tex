\section{Experiments}
\label{sec:experiments}

We implemented three showcase applications which we plan to launch together with \tc.
We provide a brief description of our applications followed by cost and performance measurements.
We refer the reader to Appendix~\ref{sec:applicationsfull} for more details on the applications and code samples.

\section{Applications: Requesting Contracts}

\paragraph{Financial derivative ({\sf CashSettledPut}).} Financial derivatives are one of the most commonly cited applications of Ethereum, and exemplify the need for a data feed on financial instruments (specifically, stocks). We implemented an example contract {\sf CashSettledPut} for what is called a {\em cash-settled put option}. This an offer to sell an asset (stock) at an agreed price on or before a particular expiration date (in anticipation of a price drop). It is ``cash-settled'' in that the sale is implicit, i.e., no stock changes hands, only cash reflecting the stock's value. In our implementation, the issuer of the option specifies a strike price $P_S$, expiration date, unit price $P_U$, and maximum number of units $M$ she is willing to sell. A customer may send a request to the contract specifying the number $X$ of option units to be purchased and containing the associated fee ($X \times P_U$). A customer may then exercise the option by sending another request prior to the expiration date. {\sf CashSettledPut} calls \tc to retrieve the closing price $P_C$ of the underlying instrument on the day the option was exercised, and pays the customer $X \times (P_S - P_C)$. To ensure sufficient funding to pay out, the contract must be endowed with ether value at least $M \times P_S$.

\iffalse
\paragraph{Financial Derivatives.}  In order to implement a financial derivative as a smart contract, we require information about the corresponding financial instrument upon which the derivative depends (typically a stock).  As an example, we implemented a cash-settled put option.  The issuer of the option creates a contract for a particular stock, strike price, time period, unit price, and the maximum number of units he is willing to sell.  Customers may purchase the option by sending requests to the contract along with the associated fee indicating the number of units of the option they would like to buy.  Until the expiration date, customers may choose to exercise the put option by making another request to the option contract.  The contract then requests that TC retrieve the closing price of the underlying instrument on the day the option was exercised, and pays out to the customer the difference between the strike price and the closing price for each unit of the option purchased.  To ensure the contract always has sufficient funds to pay out, it must control value of at least the strike price times the maximum number of units sold.
\fi

\paragraph{Flight insurance ({\sf FlightIns}).} Flight insurance indemnifies a purchaser should her flight be delayed or canceled. We have implemented a simple flight insurance contract called {\sf FlightIns}. Our implementation showcases \tc's private-datagram feature to address an obvious concern:  That customers may not wish to reveal their travel plans publicly on the blockchain. 

An insurer stands up {\sf FlightIns} with a specified policy fee, payout, and lead time $\Delta T$. ($\Delta T$ is set large enough to ensure that a customer can't anticipate flight cancellation or delay due to weather, etc.) To purchase a policy, a customer sends the {\sf FlightIns} a ciphertext  $C$ under the \tc's pubic key $\pkTC$ of the ICAO flight number $FN$ and scheduled time of departure $T_D$ for her flight, along with the policy fee. {\sf FlightIns} sends \tc a private-datagram request containing the current time $T$ and the ciphertext $C$. \tc decrypts $C$ and checks that the lead time meets the policy requirement, i.e., that $T_D - T \geq \Delta T$. \tc then scrapes a flight information data source at time $T_D$ to check the flight status, and returns to {\sf FlightIns} predicates on whether the lead time was valid and whether the flight has been delayed or cancelled. If both predicates are true, then {\sf FlightIns} returns the payout to the customer. Note that $FN$ is never exposed in the clear.

Despite the use of private datagrams, {\sf FlightIns} as described here still poses a privacy risk, as the {\em timing} of the predicate delivery by \tc leaks information about $T_D$, which may be sensitive information; this, and the fact that the payout is publicly visible, could also indirectly reveal $FN$. {\sf FlightIns} addresses this issue by including in the private datagram request another parameter $t > T_D$ specifying the time at which predicates should be returned. By randomizing $t$ and making $t - T_D$ sufficiently large, {\sf FlightIns} can substantially reduce the leakage of timing information. 

\iffalse
\paragraph{Flight insurance.} Flight insurance, which provides a payout to the purchaser in the event that their flight is delayed or canceled, is a particularly interesting application of smart contracts as it requires  the transmission of data that the customer may wish to keep private, such as the flight number, via the blockchain.  In order to maintain data privacy, TC supports encrypted datagram requests in which customers encrypt sensitive query information under a public key held by the \encname prior to publicly posting it to the blockchain.  \encname code is then responsible for decrypting and carrying out the query and passing a result back to the requesting contract stripped of private information.  

In the case of the flight insurance contract we implemented, an insurance provider creates an insurance contract with a specified fee and payout in ether that serves up to $2^{64}$ requests.  A customer who wishes to buy insurance for his flight first encrypts the ICAO flight number and scheduled time of departure for his flight, and attaches the ciphertext to a transaction he submits to the insurance contract.  If the transaction has value in ether equivalent to the fee, the contract logs the request and submits the provided ciphertext as a request to TC.  \encname code then decrypts and processes the request, determines if the specified flight was delayed or canceled, and returns a response to the insurance contract indicating whether or not the contract should pay out for a particular request.  Note that there is no flight information contained in this response.

However, the publicly visible outcome of the contract may leak information about which flight a particular customer was on, especially if the customer received a payout for flight cancellation.  An adversary attempting to compromise privacy may then narrow his search to a list of recently canceled flights.  This can be partially solved by including two encrypted addresses in the request, one owned by the customer and one owned by the insurance provider.  The \encname passes back to the requesting contract the customer-owned address if the flight is canceled, and the provider-owned address otherwise.  The contract then makes the payout to the returned address, and the adversary gains no new information from the payment so long as he cannot distinguish between the two addresses.
\fi

\paragraph{Steam Marketplace.} Steam \kyle{reference?} is an online gaming platform that supports thousands of games and maintains its own marketplace, where users can trade, buy, and sell games and other virtual items.  Through the Steam trading API, for which a key is issued to each user, we can construct a contract that implements the sale of games and items for ether using custom datagrams.  A user wishing to sell items creates a contract specifying the items to be sold along with a price in ether for each.  A user wishing to buy the items creates a Steam trade offer requesting the items (which the seller must accept out of band through either a Steam client or the Steam API), and then submits an Ethereum transaction with value in ether equal to the specified price along with an attached ciphertext containing a reference to the trade offer and his Steam API key.  The API key of either the buyer or the seller is required in order to view the contents of the trade.  The contract submits a request to TC using the provided ciphertext, and relies on TC to verify the contents and status of the trade and return the result.  If the trade was successfully accepted by the seller and the items transferred to the buyer, then the contract transfers the buyer's ether to the seller's account.  Otherwise if the trade is unsuccessful, the buyer's ether is refunded by the contract.

There is a clear parallel between the exchange of virtual goods for ether and the exchange of fiat currency for ether.  The contract remains mostly the same; virtual goods are simply replaced with dollars and the Steam API is substituted out for a (preferably read-only) API for a user's bank statements.  In both cases, the \encname must be trusted not to compromise the user's privacy (or worse if the provided API keys have additional privileges) when given access to their account statements.

%Discuss flight insurance as an example: We'd like to conceal the flight number and date. We might also want to conceal payment, so TC might ingest encrypted addresses and mix them internally.

%Micro-loans too? Linkage to Facebook / Keybase.io



\subsection{Measurements}
\label{sec:experiments}

\begin{table*}
\resizebox{\linewidth} {!}{
\begin{tabular}{l|lllll|lllll|lllll}
\toprule
& \multicolumn{5}{c|}{\sf CashSettledPut} &
  \multicolumn{5}{c|}{\sf FlightIns} &
  \multicolumn{5}{c}{\sf SteamTrade} \\
    & \textbf{mean} & \% & $t_{\max}$ & $t_{\min}$ & $\sigma_t$ & \textbf{mean}
    & \% & $t_{\max}$ & $t_{\min}$ & $\sigma_t$ & \textbf{mean} & \% & $t_{\max}$
    & $t_{\min}$ & $\sigma_t$\\
\midrule
    Ctx. switch & 1.00 & 0.6 & 3.12 & 0.25 & 0.31 
                & 1.23 & 0.24 & 2.94 & 0.17 & 0.32 
                & 1.17 & 0.20 & 3.25 & 0.36 & 0.35\\
    Web scraper & 157  & 87.2 & 258 & 135 & 18 
                & 482  & 95.4 & 600 & 418 & 31 
                & 576  & 96.2 & 765 & 489 & 52\\
    Sign        & 20.2 & 11.2 & 26.6 & 18.7 & 1.52 
                & 20.5 & 4.0 & 25.3 & 18.9 & 1.4 
                & 20.3 & 3.4 & 24.8 & 18.8 & 1.28\\
    Serialization 
                & 0.40 & 0.2 & 0.84 & 0.24 & 0.08 
                & 0.38 & 0.08 & 0.67 & 0.20 & 0.08 
                & 0.39 & 0.07 & 0.65 & 0.24 & 0.09\\
\midrule
\midrule
    \textbf{Total} 
                & 180 & 100 & 284 & 158 & 18 
                & 505 & 100 & 623 & 439 & 31 
                & 599 & 100 & 787 & 510 & 52 \\
\bottomrule
\end{tabular}
}
\caption{Enclave response time $t$, with profiling breakdown. All times are in {\bf milliseconds}.
We executed 500 experimental runs, and report
the statistics including 
the average ({\bf mean}), proportion (\%), maximum ($t_{\max}$),
minimum ($t_{\min}$), and standard deviation ($\sigma_t$). Note that {\bf Total} is the end-to-end response time as 
defined in \emph{Enclave Response Time}. Times may not
sum to this total due to minor unprofiled overhead.}
\label{tab:eval_profiling}
\vspace{-1em}
\end{table*}

We evaluated the performance of \tc on a Dell Inspiron 13-7359 laptop 
with an Intel i7-6500U CPU and 8.00GB memory, one of the few SGX-enabled systems commercially available at the
time of writing.
We show that on this single host---not even a server, but a consumer device---our implementation of \tc can easily process
transactions at the peak global rate of Bitcoin, currently the most heavily loaded decentralized blockchain. 

We report mean run times (with the standard deviation in parenthesis) over 100 trials.
%a popular, modern blockchain such as Bitcoin. (Two such hosts could match Bitcoin's peak global transaction rate should it be scaled up through reparametrization.) 

\paragraph{TCB Size.}
The trusted computing base (TCB) of Town Crier includes the~\encname and \tcontract. The \encname consists of approximately 46.4k
lines of C/C++ code, the vast majority of which (42.7k lines) is the modified mbedTLS library \cite{mbedtls}.
The source code of mbedTLS has been widely deployed and tested, while the
remainder of the~\encname codebase is small enough to admit formal verification.
The \tcontract is also compact;
it consists of approximately 120 lines of Solidity code.

\paragraph{Enclave Response Time.}
\label{subsec:response time}
We measured the enclave response time for handling a \tc request, defined as the interval between (1)  
the \medname sending a request to the enclave 
and (2) the \medname receiving a response from the enclave. 

Table \ref{tab:eval_profiling} summarizes the total enclave response time as
well as its breakdown over 500 runs.  For the three applications we
implemented, the enclave response time ranges from {\bf 180 ms} to {\bf 599 ms}.
The response time is clearly dominated by the web scraper time, i.e., the time
it takes to fetch the requested information from a website.  Among the three
applications evaluated, {\sf SteamTrade} has the longest web scraper time, as it interacts with the target website
over multiple roundtrips to fetch the desired datagram.


\paragraph{Transaction Throughput.}
\begin{figure}[h]
  \resizebox {\columnwidth} {!}{
\begin{tikzpicture}
\begin{axis}[
    every axis/.append style={font=\small},
    legend pos=north west,
    xlabel=Number of enclaves on a single machine,
    ylabel=Throughput (tx/sec),
    ylabel near ticks,
    legend cell align=left,
    legend style={font=\small},
    xmin=0, xmax=21,
    ymin=0, ymax=75,
]
\addplot [dashed,brown,domain=1:20, samples=20, forget plot]{4.9*x};
\addplot [dashed,blue,domain=1:20, samples=20, forget plot]{2.1*x};
\addplot [dashed,red,domain=1:20, samples=20, ]{1.60*x};
\addplot [mark=square*,red,   error bars/.cd,y dir=both,y explicit] table[x=n,y=steam-mean,y error=steam-std]{data.csv};
\addplot [mark=square*,blue,  error bars/.cd,y dir=both,y explicit] table[x=n,y=flight-mean,y error=flight-std]{data.csv};
\addplot [mark=square*,brown, error bars/.cd,y dir=both,y explicit] table[x=n,y=put-mean,y error=put-std]{data.csv};
\legend{{\sf Linear Scaling},{\sf SteamTrade},{\sf FlightIns},{\sf CashSettledPut}}
\end{axis}
\end{tikzpicture}
}
\caption{Throughput on a single SGX machine.  The x-axis is the number of
concurrent enclaves and the y-axis is the number of tx/sec. 
Dashed lines indicate the ideal scaling for each application, and error bars, the standard deviation.
We ran 20 rounds of experiments (each round processing 1000
transactions in parallel).}

\label{fig:trpt}
\end{figure}
We performed a sequence of experiments measuring the transaction throughput while scaling up the number of concurrently running enclaves 
on our single SGX-enabled host from 1 to 20. 20 \tc enclaves is the maximum possible given the enclave memory constraints on the specific machine model we used.
Fig.~\ref{fig:trpt} shows that, for the three applications evaluated,
{\bf a single SGX machine can handle
15 to 65
tx/sec}. 

Several significant data points show
how effectively \tc can serve the needs of 
today's blockchains for authenticated data: 
Ethereum currently handles under 1 tx/sec on average.
Bitcoin today handles slightly more than
3 tx/sec, and 
its maximum throughput (with full block utilization)
is roughly 7 tx/sec.
We know of no measurement study of the
throughput bound of the Ethereum  peer-to-peer network.
Recent work~\cite{blockchainscaling} indicates that Bitcoin cannot scale beyond 26 tx/sec without a protocol redesign.
Thus, with few hosts \tc can easily meet the data feed demands of  
even future decentralized blockchains.



%\begin{table*}
%\centering
%\begin{tabular}{l|r|r|r}
%\toprule
%& \multicolumn{1}{c|}{\sf CashSettledPut} &
%  \multicolumn{1}{c|}{\sf FlightIns} &
%  \multicolumn{1}{c}{{\sf SteamTrade}${}^\dagger$} \\
%\midrule
%Deliver without Cancel & 11.85\textcent & 12.00\textcent & 12.90\textcent \\ 
%Cancel arrived after Deliver & 14.70\textcent & 14.85\textcent & 15.75\textcent \\ 
%Cancel without Deliver & 13.95\textcent & 14.10\textcent & 15.00\textcent \\ 
%\bottomrule
%\end{tabular}
%\caption[caption]{\emph{Callback-independent} portion of gas costs in USD.
%  The applications differ because their input parameters are of different lengths.
%  We omit the cost of \dgcallback in the first two rows since it would necessary even if data acquisition were free.
%\\\hspace{\textwidth}
%${}^\dagger$ These numbers are for 1 item. Each additional item costs an additional 0.19\textcent.
%}
%\label{tbl:eval_gas}
%\end{table*}

\paragraph{Gas Costs.}
Currently 1 gas costs $5 \times10^{-8}$ Ether, so at the exchange rate of \$15 per Ether, \$1 buys 1.3 million gas.
Here we provide costs for our implementation components.

%Table~\ref{tbl:eval_gas} shows gas costs for calling \tc in our example applications; these costs are callback-independent to reflect datagram costs only, not application costs. We see an effective gas cost for \tc of roughly 10 to 15 cents.

The callback-independent portion of {\bf Deliver} costs about \num[group-separator={,}]{35000} gas (2.6\textcent), so this is the value of $\constgasmin$.
We set $\constgasmax = \num[group-separator={,}]{3100000}$ gas (\$2.33), as this is approximately Ethereum's maximum {\tt GASLIMIT}.
The cost for executing {\bf Request} is approximately \num[group-separator={,}]{120000} gas (9\textcent) of fixed cost, 
plus \num[group-separator={,}]{2500} gas (0.19\textcent) for every 32 bytes of request parameters.
The cost to execute {\bf Cancel} is 62500 gas (4.7\textcent)
including the gas cost $\constgasinvokecancel$ and the refund $\constgascancel$ paid to \tc should {\bf Deliver} be called after {\bf Cancel}.

The total callback-independent cost of acquiring a datagram from \tc (i.e., the cost of the datagram, not the application)
ranges from 11.9\textcent\ ({\sf CashSettledPut}) to 12.9\textcent\ ({\sf SteamTrade})\footnote{This cost is for 1 item. Each additional item costs 0.19\textcent.}.
The variation results from differing parameter lengths.





\paragraph{Component-Compromise Resilience.}
\label{subsec:hedging}
For the {\sf CashSettledPut} application, 
we implemented and evaluated two modes of majority voting (as in Section~\ref{subsec:enhanced_robustness}):
\begin{itemize}[leftmargin=3mm]
  \setlength{\itemsep}{2pt}
  \setlength{\parskip}{0pt}
  \setlength{\parsep}{0pt}
\item
2-out-of-3 majority voting within the enclave, providing robustness
against data-source compromise. 
In our experiments
the enclave performed simple sequential scraping of current stock prices 
from three different data sources: Bloomberg, Google Finance and Yahoo Finance.
The enclave response time is roughly
1743 (109) ms in this case ({\it c.f.}  
1058 (88), 423 (34) and 262 (12) ms for 
each respective data source). There is no change in gas cost, as voting is done
inside the SGX enclave.
In the future, we will investigate parallelization of SGX's thread mechanism, with careful consideration of the security implications.


\item
2-out-of-3 majority voting within the requester contract,
which provides robustness against 
SGX compromise.
We ran three instances of SGX enclaves, all scraping
the same data source.  
In this scenario 
the gas cost would increase by a factor of 3 plus an additional 5.85\textcent.
So {\sf CashSettledPut} would cost 35.6\textcent\ for Deliver without Cancel.
The extra 5.85\textcent\ is the cost to store votes until a winner is known.
\end{itemize}


\paragraph{Offline Measurements.}
Recall that an enclave requires a one-time setup operation that involves attestation generation.
Setting up the \tc\ \encname takes
49.5 (7.2) ms
and attestation generation takes 
61.9 (10.7) ms, including
7.65 (0.97) ms for the report, 
and 54.9 (10.3) ms for the quote.

Recall also that since clock() yields only relative time in SGX, \tc's absolute clock is calibrated through an externally furnished wall-clock timestamp.
A user can verify the correctness of the \encname absolute clock by requesting a digitally signed timestamp.
This procedure is, of course, accurate only to within its end-to-end latency. Our experiments show that the time between \medname transmission of a clock calibration request to the enclave and receipt of a response is
11.4 (1.9) ms of which 10.5 (1.9) ms is to 
sign the timestamp.
To this must be added the wide-area network roundtrip latency, rarely more than a few hundred milliseconds. %Note too that the user can measure the the latency of its timestamp verification, and therefore determine the accuracy of its measurement.
