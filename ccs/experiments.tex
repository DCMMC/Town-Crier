\section{Experiments}
\label{sec:experiments}

\begin{table*}
\resizebox{\linewidth} {!}{
\begin{tabular}{l|lllll|lllll|lllll}
\toprule
& \multicolumn{5}{c|}{\sf CashSettledPut} &
  \multicolumn{5}{c|}{\sf FlightIns} &
  \multicolumn{5}{c}{\sf SteamTrade} \\
    & \textbf{mean} & \% & $t_{\max}$ & $t_{\min}$ & $\sigma_t$ & \textbf{mean}
    & \% & $t_{\max}$ & $t_{\min}$ & $\sigma_t$ & \textbf{mean} & \% & $t_{\max}$
    & $t_{\min}$ & $\sigma_t$\\
\midrule
    Ctx. switch & 1.00 & 0.6 & 3.12 & 0.25 & 0.31 
                & 1.23 & 0.24 & 2.94 & 0.17 & 0.32 
                & 1.17 & 0.20 & 3.25 & 0.36 & 0.35\\
    Web scraper & 157  & 87.2 & 258 & 135 & 18 
                & 482  & 95.4 & 600 & 418 & 31 
                & 576  & 96.2 & 765 & 489 & 52\\
    Sign        & 20.2 & 11.2 & 26.6 & 18.7 & 1.52 
                & 20.5 & 4.0 & 25.3 & 18.9 & 1.4 
                & 20.3 & 3.4 & 24.8 & 18.8 & 1.28\\
    Serialization 
                & 0.40 & 0.2 & 0.84 & 0.24 & 0.08 
                & 0.38 & 0.08 & 0.67 & 0.20 & 0.08 
                & 0.39 & 0.07 & 0.65 & 0.24 & 0.09\\
\midrule
\midrule
    \textbf{Total} 
                & 180 & 100 & 284 & 158 & 18 
                & 505 & 100 & 623 & 439 & 31 
                & 599 & 100 & 787 & 510 & 52 \\
\bottomrule
\end{tabular}
}
\caption{Enclave response time $t$, with profiling breakdown. All times are in {\bf milliseconds}.
We executed 500 experimental runs, and report
the statistics including 
the average ({\bf mean}), proportion (\%), maximum ($t_{\max}$),
minimum ($t_{\min}$) and standard deviation ($\sigma_t$). Note that {\bf Total} is the end-to-end response time as 
defined in Section~\ref{subsec:response time}. Times may not
sum to this total due to minor unprofiled overhead.}
\label{tab:eval_profiling}
\end{table*}

We evaluated the performance of \tc on a Dell Inspiron 13-7359 laptop 
with an Intel i7-6500U CPU and 8.00GB memory, one of the few SGX-enabled systems commercially available at the
time of writing. We
show that on this single host---not even a server, but a consumer device---our implementation of \tc can easily process
transactions at the peak global rate of Bitcoin, currently the most heavily loaded decentralized blockchain. 
%a popular, modern blockchain such as Bitcoin. (Two such hosts could match Bitcoin's peak global transaction rate should it be scaled up through reparametrization.) 

\paragraph{TCB Size.}
The trusted computing base (TCB) of Town Crier includes the~\encname and \tcontract. The~\encname consists of approximately 46.4k
lines of C/C++ code, the vast majority of which (42.7k lines) is the modified mbedTLS~\cite{mbedtls}
library. The source code of mbedTLS has been widely deployed and tested, while the
remainder of the~\encname codebase is small enough to admit formal verification.
The \tcontract is also compact: It consists of approximately 120 lines of Solidity code.

\paragraph{Enclave Response Time.}
\label{subsec:response time}
We measured the enclave response time for handling a \tc request, defined as the interval between (1)  
the \medname sending a request to the enclave 
and (2) the \medname receiving a response from the enclave. 

Table \ref{tab:eval_profiling} summarizes the total enclave response time as
well as its breakdown over 500 runs.  For the three applications we
implemented, the enclave response time ranges from {\bf 180 ms} to {\bf 599 ms}.
The response time is clearly dominated by the web scraper time, i.e., the time
it takes to fetch the requested information from a website.  Among the three
applications evaluated, {\sf SteamTrade} has the longest web scraper time, as it interacts with the target website
over multiple roundtrips to fetch the desired datagram.


\paragraph{Transaction Throughput.}
\begin{figure}[h]
  \resizebox {\columnwidth} {!}{
\begin{tikzpicture}
\begin{axis}[
    every axis/.append style={font=\small},
    legend pos=north west,
    xlabel=Number of enclaves on a single machine,
    ylabel=Throughput (tx/sec),
    ylabel near ticks,
    legend cell align=left,
    legend style={font=\small},
    xmin=0, xmax=21,
    ymin=0, ymax=75,
]
\addplot [dashed,brown,domain=1:20, samples=20, forget plot]{4.9*x};
\addplot [dashed,blue,domain=1:20, samples=20, forget plot]{2.1*x};
\addplot [dashed,red,domain=1:20, samples=20, ]{1.60*x};
\addplot [mark=square*,red,   error bars/.cd,y dir=both,y explicit] table[x=n,y=steam-mean,y error=steam-std]{data.csv};
\addplot [mark=square*,blue,  error bars/.cd,y dir=both,y explicit] table[x=n,y=flight-mean,y error=flight-std]{data.csv};
\addplot [mark=square*,brown, error bars/.cd,y dir=both,y explicit] table[x=n,y=put-mean,y error=put-std]{data.csv};
\legend{{\sf Linear Scaling},{\sf SteamTrade},{\sf FlightIns},{\sf CashSettledPut}}
\end{axis}
\end{tikzpicture}
}
\caption{Throughput on a single SGX machine.  The x-axis is the number of
concurrent enclaves and the y-axis is the number of tx / sec. 
Dashed lines indicate the ideal scaling for each application, and error bars, the standard deviation.
We ran 20 rounds of experiments (each round processing 1000
transactions in parallel).}

\label{fig:trpt}
\end{figure}
We performed a sequence of experiments measuring the transaction throughput while scaling up the number of concurrently running enclaves 
on our single SGX-enabled host from 1 to 20. 20 \tc enclaves is the maximum possible given the enclave memory constraints on the specific machine model we used.
Figure \ref{fig:trpt}
shows that for the three applications evaluated,
{\bf a single SGX machine can handle
15 to 65
tx/sec}. 

Several significant data points show
how effectively \tc can serve the needs of 
today's blockchains for authenticated data: 
Ethereum currently handles 
$< 1$ tx/sec on average. 
Bitcoin today handles slightly more than
3 tx/sec, and 
its maximum throughput (with full block utilization)
is roughly 7 tx/sec.
We know of no measurement study of the
throughput bound of the Ethereum  peer-to-peer network.
It has been shown that without
a protocol redesign, however, the current 
Bitcoin network cannot scale via reparametrization beyond  
27 tx/sec~\cite{blockchainscaling}.
Thus, with one or at most a few hosts, \tc can easily meet the data feed demands of  
even future decentralized blockchains.



%\begin{table*}
%\centering
%\begin{tabular}{l|r|r|r}
%\toprule
%& \multicolumn{1}{c|}{\sf CashSettledPut} &
%  \multicolumn{1}{c|}{\sf FlightIns} &
%  \multicolumn{1}{c}{{\sf SteamTrade}${}^\dagger$} \\
%\midrule
%Deliver without Cancel & 11.85\textcent & 12.00\textcent & 12.90\textcent \\ 
%Cancel arrived after Deliver & 14.70\textcent & 14.85\textcent & 15.75\textcent \\ 
%Cancel without Deliver & 13.95\textcent & 14.10\textcent & 15.00\textcent \\ 
%\bottomrule
%\end{tabular}
%\caption[caption]{\emph{Callback-independent} portion of gas costs in USD.
%  The applications differ because their input parameters are of different lengths.
%  We omit the cost of \dgcallback in the first two rows since it would necessary even if data acquisition were free.
%\\\hspace{\textwidth}
%${}^\dagger$ These numbers are for 1 item. Each additional item costs an additional 0.19\textcent.
%}
%\label{tbl:eval_gas}
%\end{table*}

\paragraph{Gas Costs.}
Currently 1 gas costs $5 \times10^{-8}$ Ether, so at the exchange rate of \$15 for 1 Ether, \$1 buys 1.3 million gas.
Here we provide costs for our implementation components.

%Table~\ref{tbl:eval_gas} shows gas costs for calling \tc in our example applications; these costs are callback-independent to reflect datagram costs only, not application costs. We see an effective gas cost for \tc of roughly 10 to 15 cents.

The callback-independent portion of {\bf Deliver} costs about \num[group-separator={,}]{35000} gas (2.6\textcent), so this is the value of $\constgasmin$.
We set $\constgasmax = \num[group-separator={,}]{3100000}$ gas (\$2.33), as this is approximately Ethereum's maximum {\tt GASLIMIT}.
The cost for executing {\bf Request} is approximately \num[group-separator={,}]{120000} gas (9\textcent) of fixed cost, 
plus \num[group-separator={,}]{2500} gas (0.19\textcent) for every 32 bytes of request parameters.
The cost to execute {\bf Cancel} is 62500 gas (4.7\textcent)
including the gas cost $\constgasinvokecancel$ and the refund $\constgascancel$ paid to \tc should {\bf Deliver} be called after {\bf Cancel}.

The total callback-independent cost of acquiring a datagram from \tc (i.e., the cost of the datagram, not the application)
ranges from 11.9\textcent\ ({\sf CashSettledPut}) to 12.9\textcent\ ({\sf SteamTrade})\footnote{This cost is for 1 item. Each additional item costs an additional 0.19\textcent.}.
The variation results from differing parameter lengths.





\paragraph{Component-Compromise Resilience.}
\label{subsec:hedging}
For the {\sf CashSettledPut} application, 
we implemented and evaluated two modes of majority voting (as in Section~\ref{subsec:enhanced_robustness}):
\begin{itemize}[leftmargin=3mm]
  \setlength{\itemsep}{2pt}
  \setlength{\parskip}{0pt}
  \setlength{\parsep}{0pt}
\item
2-out-of-3 majority voting within the enclave, providing robustness
against data-source compromise. 
In our experiments
the enclave performed simple sequential scraping of current stock prices 
from three different data sources: Bloomberg, Google Finance and Yahoo Finance.
The enclave response time is roughly
1743(109) ms in this case ({\it c.f.}  
1058(88), 423(34) and 262(12) ms for 
each respective data source). There is no change in gas cost, as voting is done
inside the SGX enclave.
In the future, we will investigate parallelization of SGX's thread mechanism, with careful consideration of the security implications.


\item
2-out-of-3 majority voting within the requester contract,
which provides robustness against 
SGX compromise.
We ran three instances of SGX enclaves, all scraping
the same data source.  
In this scenario 
the gas cost would increase by a factor of 3 plus an additional 5.85\textcent.
So {\sf CashSettledPut} would cost 35.6\textcent\ for Deliver without Cancel.
The extra 5.85\textcent\ is a storage cost: The requester contract must store votes
until a winner is known.
\end{itemize}


\paragraph{Offline Measurements.}
Recall that a one-time setup operation is required 
for an enclave, and involves attestation generation. We report mean run times (with the S.D. in parentheses) for these experiments over 100 trials.
Setting up the \tc~\encname takes
49.5 (7.2) ms,
and attestation generation takes 
61.9 (10.7) ms, including
7.65 (0.97) ms for report generation, 
and 54.9 (10.3) ms for quote generation.

Recall also that since clock() yields only relative time in SGX, \tc's absolute clock is calibrated through an externally furnished wall-clock timestamp.
A user can verify the correctness of the \encname absolute clock by requesting a digitally signed timestamp.
This procedure is, of course, accurate only to within its end-to-end latency. Our experiments show that the time between \medname transmission of a clock calibration request to the enclave and receipt of a response is
11.4 (1.9) ms of which 10.5 (1.9) ms is to 
sign the timestamp.
To this must be added the wide-area network roundtrip latency, at most typically a few hundred milliseconds. %Note too that the user can measure the the latency of its timestamp verification, and therefore determine the accuracy of its measurement.
