\section{Future Work}
\label{sec:future-work}

We plan to develop \tc after its initial deployment to incorporate a number of additional features.
Here we discuss the a few of those features here.

\paragraph{Freeloading Protection.}
There are concerns in the Ethereum community about ``parasite contracts'' that forward or resell datagrams from fee-based data feeds \cite{parasite}.
In order to guard against these parasites, we plan to deploy a novel mechanism in \tc.
All users $U$ of a requesting contract would generate keys for a multi-party signature and submit the private keys to \tcont encrypted under \pkTC.
\tc will then produce a datagram signed with the combination of all supplied keys.

With all of the associated secret keys, \tcont can produce a datagram with the appropriate signature and each user $U_i$ can be sure the datagram is valid since $U_i$ did not collude in its creation.
However, potential parasitic users cannot determine whether the datagram was produced by \tcont or by $U$, and thus whether or not it is valid.
Such a \emph{source-equivocal datagram} renders parasite contracts less trustworthy and thus less attractive.

\paragraph{Revocation Support.}
There are two forms of revocation relevant to \tc.
First, the certificates of data sources may be revoked.
Since \tc already uses HTTPS, it could easily use an Online Certificate Status Protocol (OCSP) to check TLS certificates.
Second, an SGX host could become compromised, prompting revocation of its EPID signatures by Intel.
The Intel Attestation Service (IAS) will reportedly provide a means of disseminating such revocations.
Conveniently, clients already use the IAS when checking the attestation \sigatt, so this requires no modification to \tc.

\paragraph{Hedging Against SGX Compromise.}
We discussed in Section~\ref{subsec:enhanced_robustness} how \tc can support majority voting across SGX hosts and data sources.
Design enhancements to \tc could reduce associated latency and gas costs.
For SGX voting, we plan to investigate a scheme in which SGX-enabled \tc hosts agree on a datagram value $X$ via Byzantine consensus.
The hosts may then use a threshold digital signature scheme to sign the datagram response from \tcadd,
and each participating host can monitor the blockchain to ensure delivery.

\paragraph{Generalized Custom Datagrams.}
In our {\sf SteamTrade} example contract we demonstrated a custom datagram that scrapes an individual user's online account using her credentials.
A more general approach would allow users to specify their own funconalities as general purpose code and achieve inexpensive data-source-enriched emulation of private contracts in Hawk \cite{hawk}.
Furnishing large custom datagrams on the blockchain would be prohibitively expensive, but code could be easily loaded off-chain.
Of course, a system that deploys arbitrary user code raises many security and confidentiality concerns which \tc would need to address.
