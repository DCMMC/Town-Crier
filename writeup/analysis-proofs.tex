
\section{Proofs of Security Analysis}
\label{sec:analysis proofs}

This section contains the proofs of the theorems we stated in Section~\ref{sec:analysis}


\begin{proof}[Proof of Gas neutrality for Town Crier (sketch)]
Honest relay means $\resumecall(\dgid, \dgform)$ will always be legitimate and $\dgform = \dgform'$ in {\bf Deliver}.
While $\fee$ is specified by a potentially malicious user, \tcont rejects the request unless $\constgasmin \leq \fee \leq \constgasmax$.
This means $\gasdeliver = \constgasmax \geq \fee$.
We also will never respond to the same request twice (these are specified in the enclave protocol).
This means that ${\sf isDelievered}[\dgid]$ will always be unset and we will always successfully retrieve a stored tuple and not abort.
From here we consider two cases.

\paragraph{Deliver called before Cancel.}
In this case {\bf Deliver} calls $\dgcallback$ with a gas limit $\gascallback$ of $\fee - \constgasmin$ (which must be positive because of the check in {\bf Request}).
$\constgasmin$ is set so that it is no less than the gas cost of {\bf Deliver} not including $\dgcallback$.
That means that the total gas spent for {\bf Deliver} will be no greater than $\constgasmin + (\fee - \constgasmin) = \fee$, which is exactly what $\tcadd$ is sent for reimbursement.
Moreover, the gas limit provided to {\bf Deliver} must be high enough because $\fee \leq \constgasmax = \gasdeliver$, so we cannot run out of gas.
Finally, because $\fee$ was deposited with {\bf Request} and neither {\bf Deliver} nor {\bf Cancel} have been called before for this request, there must be at least $\fee$ stored $\tcont$.

\paragraph{Deliver called after Cancel.}
In this case $\tcadd$ is sent $\constgascancel$ and performs only operations of known cost.
Because $\constgascancel$ is explicitly set to be the cost of those operations, this reimburses $\tcadd$ properly.
Since $\fee$ was deposited with {\bf Request}, the first call to {\bf Cancel} leaves $\constgasmax$ associated with that request and subsequent calls do nothing,
and {\bf Deliver} has not been called before for this request, there must be at least $\constgasmax$ left in $\tcont$ to send.
\end{proof}





\begin{proof}[Proof of Fair Expenditure for Honest Requester (sketch)]
We consider the cases when the request is delivered and when it is not.

\paragraph{Request Delivered.}
If the request is delivered, then we know by Theorem \ethan{refer} it is an authentic datagram corresponding to a valid request.
In this case an honest user pays $\gasrequest$ in gas to execute {\bf Request} and $\fee$ in money to reimburse \tc for {\bf Deliver}.
By Ethereum's gas model, $\gasrequest$ is exactly the cost of computation for {\bf Request}.
If we let $\gascallback$ be the gas required to execute $\dgcallback$, then an honest user will set $\fee = \gascallback + \constgasmin$.
By definition, $\constgasmin$ is exactly the cost of executing {\bf Deliver} not including $\dgcallback$, so therefore $\fee$ is exactly the cost to deliver a datagram.
\tcs is only reimbursed all of $\fee$ if the datagram is successfully authenticated and delivered, so in this case the requester's expenses total exactly the cost of the computation requested by the user and executed on her behalf.

An honest user may believe the request will not be delivered can call {\bf Cancel} at the same time {\bf Deliver} is called, but have {\bf Deliver} be executed first.
In this case the user will not receive a refund from {\bf Cancel} and will pay $\gascancel$ to execute the computation,
but this is the same as the above case except the user has requested and paid a {\bf Cancel} operation and no more.


\paragraph{Request Not Delivered.}
If the request is not delivered, an honest user will eventually call {\bf Cancel} for the request (and will do so only once).
In this case, the user pays $\gasrequest$ for {\bf Request}, $\gascancel$ for {\bf Cancel}, submits $\fee = \gascallback + \constgasmin$ to {\bf Request},
and is refunded $\fee - \constgascancel$ by {\bf Cancel}.

The computation that was actually executed on the user's behalf was that for {\bf Request} and {\bf Cancel}.
A completely fair payment would thus require the user to pay $\gasrequest + \gascancel$.
In our protocol, the user's actual net expense is
\begin{align*}
  \gasrequest + \gascancel & + \fee - (\fee - \constgasmin) \\
                           & = \gasrequest + \gascancel + \constgasmin.
\end{align*}
Thus the user's expense has exceeded the computation executed on her behalf by exactly $\constgasmin$.
\end{proof}

