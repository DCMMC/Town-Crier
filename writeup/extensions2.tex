
\section{Handling gas as implemented}

\ethan{This section is a writeup of the things that need to change to make sure that protocol with fees conforms more precisely to the implementation.}

\subsection{Handling Transaction Fees}

To mitigate potential Denial-of-service (DoS) attacks, Ethereum employs a fee mechanism, referred to as ``gas'',
where the user requesting a transaction (that can invoke an entry point to a contract) pays a transaction fee.
The fee is proportional to the computation required to execute the specified transaction.


\paragraph{Notations and assumed execution model.}
In Figure \ref{tbl:tc-contract2}, we use the notation $\gas$ to denote transaction fees (i.e. gas) where \$ is a type annotation of {\sf G} denotes the numerical amount of gas.
Non-gas currency is denoted $\fee$ where \$ is again a type annotation and {\tt F} denotes the amount.
For simplicity, our notation assumes that gas and normal currency adopt the same units.

We assume the blockchain contract adopts the following execution model which closely resembles that of Ethereum.
This model is also described graphically in Figure \ref{fig:gas-model}.

\begin{itemize}[leftmargin=1.5em]
  \item {\it Providing gas.}
    When a transaction is submitted, it invokes an entry point in the contract.
    The transaction submitter provides a gas amount to activate the entry point.

  \item {\it Extra gas.}
    If a transaction includes more gas than is necessary to process that transaction,
    \emph{all excess gas is refunded to the caller as part of the transaction}.

  \item {\it Gas exhaustion.}
    Consider the entry point of a contract as a function.
    If a function calls another function, the parent may limit the gas available to the child.
    This limit may not exceed the remaining gas available to the parent.
    If no limit is specified, all remaining gas is available to the child.
    When a function cannot be executed to completion within the specified gas limit,
    the function execution is aborted and all state is reverted to immediately preceding that function call.
    Note that if the original entry point runs out of gas, the entire transaction is aborted.

\end{itemize}


\begin{figure}
\centering
\begin{tikzpicture}
  [local-entity/.style={entity,minimum height=3.5em,text width=8em}]
  \node[local-entity,trusted] (ctc) {};
  \node[local-entity,draw=none,anchor=north] (ctc-inner) at (ctc.north) {TC Contract\\$\tcont$};
  \node[local-entity,trusted,right=6em of ctc,text width=5em] (enc) {Enclave};
  \node[local-entity,fill=white,below=3.5em of ctc] (cu) {User Contract\\$\reqcont$};
  \node[local-entity,fill=red!30,below=2.5em of cu] (user) {User};

  \path[-stealth,color=red,ultra thick] (user) edge [transform canvas={xshift=-3.5em}] ([yshift=0.75em]cu.south);
  \path[color=red,thick] (cu.south) edge [right,transform canvas={xshift=-3.5em}] node [text=black,text width=4em,align=center,yshift=1.75em] {\footnotesize Insufficient\\[-0.25em]gas aborts\\[-0.2em]with no\\[-0.5em]effect} (ctc.south);
  \node[draw=red,cross out,minimum size=1ex,thick] () at ([xshift=-3.5em]ctc.south) {};

  \draw[-stealth,color=green!50!black,line width=0.8ex] ([yshift=-0.25em]enc.west) -- ([xshift=-1.2em,yshift=-0.25em]ctc.east);
  \draw[-stealth,color=green!50!black,line width=0.5ex] ([yshift=-0.25em]enc.west) -| ([xshift=3em,yshift=-0.9em]cu.north);
  \draw[color=green!50!black,thick,dashed]  ([xshift=3.75em]cu.north) |- ([yshift=-1.25em]ctc.east);
  \path[-stealth,color=green!50!black,ultra thick,dashed]  ([yshift=-1.25em]ctc.east) edge [below] node [text=black,text width=4em,align=center,xshift=-0.4em] {\footnotesize Extra gas\\[-0.4em]refunded} ([xshift=0.8em,yshift=-1.25em]enc.west);

  \begin{pgfonlayer}{background}
    \node[bg-box,
          blockchain-color,
          fit={($(ctc.north west)+(-0.5em,0.5em)$)($(cu.south east)+(0.5em,-0.5em)$)},
          label=above:{\bf Blockchain}] () {};
    \node[bg-box,
          tc-server-color,
          fit={($(enc.north east)+(0.5em,0.5em)$)($(enc.south west)+(-0.5em,-0.5em)$)},
          label=above:{\bf TC Server}] () {};
  \end{pgfonlayer}
\end{tikzpicture}
\caption{{\bf Gas Payment in Ethereum}}
\label{fig:gas-model}
\end{figure}



\begin{figure}
\begin{tabularx}{\linewidth}{|@{\hspace{3pt}}r@{\hspace{1ex}}X@{\hspace{3pt}}|}
  \hline

  \multicolumn{2}{|c|}{{\bf Town Crier blockchain contract \tcont with fees}} \\
  {\bf Request:} & On recv $({\sf params}, {\sf callback}, \gasrequest + \fee)$ from some $\reqcont$: \\
                 & Assert $\constgasmin \leq \fee \leq \constgasmax$ \\
                 & $\dgid := \text{Counter}$; \ \ $\text{Counter} := \text{Counter} + 1$ \\
                 & Store $(\dgid, \dgform, \dgcallback, \fee, \reqcont)$ \\[-0.8em]
                 & {\it \sgray{//~$\fee$ held by contract}} \\

  {\bf Deliver:} & On recv $(\dgid, \dgform, \dgm, \gasdeliver)$ from $\tcadd$: \\
                 & If ${\sf bCanceled}[{\sf id}]$ and not ${\sf bDelivered}[{\sf id}]$: \\
                 & \quad Send \constgascancel\ to \tcadd \\
                 & \quad Set ${\sf bDelivered}[\dgid]$ \\
                 & \quad Return \\
                 & Retrieve stored $(\dgid, \dgform', \dgcallback, \fee, \_)$ \\
                 & \quad \sgray{\it //~abort if not found} \\
                 & Assert $\dgform = \dgform'$ \\
                 & Assert $\fee \leq \gasdeliver$ \\
                 & Set ${\sf bDelivered}[\dgid]$ \\
                 & Send $\fee$ to \tcadd \\
                 & Call $\dgcallback(\dgm)$ with $\fee - \constgasmin$ max gas \\

  {\bf Cancel:}  & On recv $(\dgid, \gascancel)$ from \reqcont \\
                 & Retrieve stored $(\dgid, \_, \_, \fee, \reqcont')$ \\
                 & \quad \sgray{\it //~abort if not found} \\
                 & Assert $\reqcont = \reqcont'$ \\
                 & Assert ${\sf bDelivered}[\dgid]$ not set \\
                 & Assert ${\sf bCanceled}[\dgid]$ not set \\
                 & Assert $\fee \geq \constgascancel$ \\
                 & Set ${\sf bCanceled}[\dgid]$ \\
                 & Send $(\fee - \constgascancel)$ to \reqcont \\[-0.8em]
                 & {\it \sgray{//~\constgascancel\ held by contract}} \\
  \hline
\end{tabularx}
\caption{
Town Crier contract \tcont reflecting fees.
An honest requester would set $\fee$ to be the gas amount
required to execute the {\bf Deliver} entry point including $\dgcallback$.
Town Crier sets $\gasdeliver := \constgasmax$, but limits gas provided for $\dgcallback$ ensure that no more than $\fee$ is spent.
Ethereum automatically refunds any unspent gas, so Town Crier will not lose money.
If the {\bf Deliver} is called after cancellation, then at most $\constgascancel$ will be spent, which is exactly the amount retained from the requester's fee when {\bf Cancel} is called.
}
\label{tbl:tc-contract2}
\end{figure}

\begin{figure}
\centering
\begin{tikzpicture}
  [contract/.style={entity,minimum height=3.5em,text width=7.5em},
   wallet/.style={entity,minimum height=3.5em,text width=5em},
   type-box-color/.style={fill=black!10},
   blocked-out-label/.style={text=black,type-box-color,text height=0.6em}]
  \node[wallet,fill=white] (user) {User ${\cal P}_U$};
  \node[contract,fill=white,right=6em of user] (cu) {User Contract\\$\reqcont$};
  \node[wallet,trusted,below=6em of user] (tc-wallet) {$\tcadd$};
  \node[contract,trusted,right=6em of tc-wallet] (ctc) {TC Contract\\$\tcont$};
  \node[color=maroon,draw,anchor=south east] (fee) at ([xshift=-0.25em,yshift=0.25em]ctc.south east) {\small $\fee$};

  \draw[color=gold,opacity=0.35,line width=1ex] ([yshift=1.25em]user.east) -| ([xshift=3.25em]cu.south);
  \path[-stealth,color=gold,line width=1.15ex] (user) edge [transform canvas={yshift=1.25em}] (cu);
  \path[-stealth,color=gold,line width=1ex] (cu) edge [transform canvas={xshift=3.25em}] (ctc);
  \path[-stealth,color=maroon,ultra thick] (cu.south) edge [left,transform canvas={xshift=3.25em}] node [blocked-out-label,xshift=1em,yshift=1.4em] {\small \smash{$\gasrequest + \fee$}} ([yshift=-1.5em]ctc.north);
  \node[anchor=north,xshift=1em,yshift=-0.75em] () at (cu.south) {\small \bf \underline{\smash{Request}}};

  \path[-stealth,color=gold,line width=0.8ex] ([yshift=0.6em]tc-wallet.east) edge [above] node () {\small $\gasdeliver$} ([yshift=0.6em]ctc.west);
  \draw[color=gold,opacity=0.35,line width=0.5ex] ([yshift=0.6em]tc-wallet.east) -| ([xshift=-3em]ctc.north);
  \path[-stealth,color=gold,line width=0.5ex] (ctc.north) edge [transform canvas={xshift=-3em}] node [blocked-out-label,xshift=0.9em,yshift=-1.25em] {\small \smash{$\fee - \constgasmin$}} (cu.south);
  \path[-stealth,color=maroon,ultra thick] (ctc.west) edge [below,transform canvas={yshift=-0.6em}] node [xshift=0.4em] {\small $\fee$} ([xshift=-0.8em]tc-wallet.east);
%  \path[-stealth,color=gold,ultra thick,dashed] (ctc.west) edge [below,transform canvas={yshift=-1em}] node [text=black,xshift=0.4em] {\small $\gasdeliver - \fee$} ([xshift=-0.8em]tc-wallet.east);

  \path[] (tc-wallet.north east) edge [draw=none,above] node [yshift=1em] {\small \bf \underline{Deliver}} (ctc.north west);

  \begin{pgfonlayer}{background}
    \node[bg-box,
          blockchain-color,
          fit={($(ctc.south east)+(1em,-1em)$)($(user.north west)+(-1em,2em)$)},
          label=above:{\bf Blockchain}] () {};
    \node[bg-box,
          type-box-color,
          fit={($(ctc.south east)+(0.5em,-0.5em)$)($(cu.north west)+(-0.5em,0.5em)$)},
          label=above:{\bf Contracts}] () {};
    \node[bg-box,
          type-box-color,
          fit={($(tc-wallet.south east)+(0.5em,-0.5em)$)($(user.north west)+(-0.5em,0.5em)$)},
          label=above:{\bf Wallets}] () {};
  \end{pgfonlayer}
\end{tikzpicture}
\caption{{\bf Blockchain Money Flow within Town Crier.}
Red arrows are used to denote the flow of money.
Solid brown arrows show gas provided to function calls and the dashed brown arrow indicates the refund of unused gas.
}
\label{fig:money-flow}
\end{figure}




\ethan{I'm not copying Elaine's text about defining values here because I don't have significant changes.}



\paragraph{Town Crier protocol with transaction fees.}
Our basic Town Crier system implements a policy where the requester pays for all gas needed and Town Crier effectively pays nothing.
We now describe how this can be realized by modifying the fee-free protocol described in Section \ethan{refer}.

\begin{itemize}[leftmargin=1.5em]
  \item {\it Initialization.}
    We assume that Town Crier deposits at least $\constgasmax$ into the wallet $\tcadd$.

  \item {\it Town Crier blockchain contract.}
    Figure \ref{tbl:tc-contract2} describes the Town Crier blockchain contract reflecting fees.
    Since Town Crier's account $\tcadd$ has to invoke the {\bf Deliver} entry point, it advances a gas payment $\constgasmax$.
    We prove in Section \ethan{ref-gas-neutrality} that this is always enough gas and providing extra is not a problem.

  \item {\it Town Crier Relay.}
    The relay behavior does not change with the presence of fees.
    It still monitors the blockchain and whenever the contract \tcont stores a new request $(\dgid, \dgform, \_, \_, \_)$,
    it invokes $\enclaveprog$ with $\resumecall(\dgid, \dgform)$.

  \item {\it Town Crier enclave.}
    We make the following small modification to the fee-free protocol.
    Instead of signing the tuple $(\dgid, \dgform, \dgm)$ at the end of its execution,
    the enclave now signs the tuple $(\dgid, \dgform, \dgm, \gasdeliver)$ where $\gasdeliver = \constgasmax$.

  \item {\it Requester.}
    An honest requester would behave the same was as in Figure \ethan{refer} except for additionally sending $\gasrequest$ gas and $\fee$ money with each request.
    It would set $\gasrequest$ to be at least the cost of executing the {\bf Request} entry point
    and $\fee$ to be the cost of executing the {\bf Deliver} entry point (including executing the user-defined $\dgcallback$ function).

    If the honest requester did not receive a callback for some $\dgid$ (or simply no longer needs the response),
    it can invoke {\bf Cancel} with $(\dgid, \gascancel)$ where $\gascancel$ is the gas cost of executing the {\bf Cancel} entry point.
    In this case the requester will be refunded $\fee - \constgascancel$ with $\constgascancel$ withheld as a cancellation fee
    to reimburse Town Crier for the small amount of gas expected should it try to deliver a datagram after the cancellation.
\end{itemize}




\subsection{Gas Neutrality}


\begin{theorem}[Gas neutrality for Town Crier]
Assuming an honest Town Crier relay,
Town Crier's wallet account $\tcadd$ will have at least $\constgasmax$ remaining after each {\bf Deliver} call.
\end{theorem}

\begin{proof}[(sketch)]
Honest relay means $\resumecall(\dgid, \dgform)$ will always be legitimate and $\dgform = \dgform'$ in {\bf Deliver}.
Also, $\gasdeliver = \constgasmax \geq \fee$, and we will never respond to the same request twice (these are specified in the relay protocol), so we will never abort.

If ${\sf bIsCanceled}[\dgid]$, then there is at least $\constgascancel$ left in \tcont from {\bf Cancel} which is returned to $\tcadd$,
and, by design, that is enough to cover gas costs when a request is canceled.

Otherwise we limit the gas provided to $\dgcallback$ to $\fee - \constgasmin$, and $\constgasmin$ is set so that it is at least the cost of {\bf Deliver} not including $\dgcallback$.
This means that the total gas used is no greater than $\constgasmin + (\fee - \constgasmin) = \fee$, which is exactly what $\tcadd$ is sent for reimbursement.
\end{proof}


\begin{theorem}[Bounded loss for honest requester]
For any request $(\dgform, \dgcallback, \_)$ submitted by an honest user $\reqcont$, one of the following will occur:
\begin{itemize}
  \item $\dgcallback$ will be successfully invoked with a valid datagram matching the requested parameters.

  \item $\reqcont$ invokes {\bf Cancel} and loses at most $\gasrequest + \gascancel + \constgascancel$ money.
\end{itemize}
\end{theorem}

\begin{proof}[(sketch)]
If the requester is honest, then the value submitted as $\fee$ will be enough to cover delivery including $\dgcallback$.
If this is the case and {\bf Deliver} is called while $\tcadd$ has $\gasdeliver$ available (i.e. has not had a malicious relay burn money)
then the requester gets its datagram.

If the requester doesn't get its datagram but provided a sufficient value for $\fee$, the request cannot have been marked as delivered.
As long as the request remains undelivered, the requester can call {\bf Cancel}.
Cancel refunds all but $\constgascancel$ of $\fee$, so the requester had to pay $\gasrequest + \gascancel + \fee$ and was refunded $\fee - \constgascancel$, so the requesters total expenditure is
\begin{align*}
  (\gasrequest + \gascancel + \fee) - (\fee - \constgascancel& ) \\ = \gasrequest + \gascancel & + \constgascancel.
\end{align*}
\end{proof}



