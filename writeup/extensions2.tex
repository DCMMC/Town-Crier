
\section{Handling gas as implemented}

\subsection{Handling Transaction Fees}

As we described in Section \ethan{ref gas-desc}, Ethereum employs \emph{gas}, a fee mechanism, to mitigate DoS attacks.
In order to sustainably operate \tcs on Ethereum, we must account for this cost and include reimbursement fees from requesters.
Unfortunately gas can only be used to pay for transactions, so instead we develop a two-currency resource management system
and prove that it provides reasonable guarantees for both honest requesters and \tcs.


\paragraph{Execution model and notation.}

%We assume the blockchain contract adopts the following execution model which closely resembles that of Ethereum described in Section \ethan{refer}.
%
%\begin{itemize}[leftmargin=1.5em]
%  \item {\it Providing gas.}
%    When a transaction is submitted, it invokes an entry point in the contract and provides a gas amount.
%
%  \item {\it Extra gas.}
%    If the provided gas exceeds the gas necessary to process the transaction, all excess gas is refunded as part of the transaction.
%
%  \item {\it Gas exhaustion.}
%    Consider each entry point a function call.
%    If a function calls another function, the parent may limit the gas available to the child to a value no higher than the gas available to the parent.
%    When a function cannot complete execution within its gas limit,
%    that function execution is aborted and all state is reverted to immediately prior to that call.
%    Note that if the initial entry point runs out of gas, the entire transaction is aborted.
%
%\end{itemize}

We take Ethereum's gas model ask described in Section \ethan{ref gas-desc}.
We use the notation $\gas$ to denote gas and $\fee$ to denote non-gas currency.
In both cases \$ is a type annotation and the letter denotes the numerical amount.
For simplicity, our notation assumes that gas and normal currency adopt the same units (allowing us to perform arithmetic without conversion).
We use the following identifiers to denote currency and gas amounts.
%
\begin{center}
\begin{tabular}{m{0.08\columnwidth}m{0.85\columnwidth}}
  \hline
  $\fee$
  & Currency a requester deposits to refund \tcs's gas expenditure to deliver a datagram \\
  \hline
  $\gasdeliver$ $\gasrequest$
  & {\tt GASLIMIT} when invoking {\bf Deliver} and {\bf Request}, respectively \\
  \hline
  $\gascallback$
  & {\tt GASLIMIT} for $\dgcallback$ while executing {\bf Deliver} \\
  \hline
  $\constgasmin$
  & Gas required for {\bf Deliver} excluding $\dgcallback$ \\
  \hline
  $\constgasmax$
  & Maximum gas \tc can spent to invoke {\bf Deliver} \\
  \hline
\end{tabular}
\end{center}
%
Note that $\constgasmin$ and $\constgasmax$ are system constants,
$\fee$ is chosen by the requester (and may be malicious if the requester is dishonest),
and $\gasdeliver$ is chosen by the \tc~\encname when calling {\bf Deliver}.
Though $\gasrequest$ is set by the requester, we need not worry about the value.
If it is too small, Ethereum will abort the transaction and there will not be a request.


%\begin{figure}
%\centering
%\begin{tikzpicture}
%  [local-entity/.style={entity,minimum height=3.5em,text width=8em}]
%  \node[local-entity,trusted] (ctc) {};
%  \node[local-entity,draw=none,anchor=north] (ctc-inner) at (ctc.north) {TC Contract\\$\tcont$};
%  \node[local-entity,trusted,right=6em of ctc,text width=5em] (enc) {Enclave};
%  \node[local-entity,fill=white,below=3.5em of ctc] (cu) {User Contract\\$\reqcont$};
%  \node[local-entity,fill=red!30,below=2.5em of cu] (user) {User};
%
%  \path[-stealth,color=red,ultra thick] (user) edge [transform canvas={xshift=-3.5em}] ([yshift=0.75em]cu.south);
%  \path[color=red,thick] (cu.south) edge [right,transform canvas={xshift=-3.5em}] node [text=black,text width=4em,align=center,yshift=1.75em] {\footnotesize Insufficient\\[-0.25em]gas aborts\\[-0.2em]with no\\[-0.5em]effect} (ctc.south);
%  \node[draw=red,cross out,minimum size=1ex,thick] () at ([xshift=-3.5em]ctc.south) {};
%
%  \draw[-stealth,color=green!50!black,line width=0.8ex] ([yshift=-0.25em]enc.west) -- ([xshift=-1.2em,yshift=-0.25em]ctc.east);
%  \draw[-stealth,color=green!50!black,line width=0.5ex] ([yshift=-0.25em]enc.west) -| ([xshift=3em,yshift=-0.9em]cu.north);
%  \draw[color=green!50!black,thick,dashed]  ([xshift=3.75em]cu.north) |- ([yshift=-1.25em]ctc.east);
%  \path[-stealth,color=green!50!black,ultra thick,dashed]  ([yshift=-1.25em]ctc.east) edge [below] node [text=black,text width=4em,align=center,xshift=-0.4em] {\footnotesize Extra gas\\[-0.4em]refunded} ([xshift=0.8em,yshift=-1.25em]enc.west);
%
%  \begin{pgfonlayer}{background}
%    \node[bg-box,
%          blockchain-color,
%          fit={($(ctc.north west)+(-0.5em,0.5em)$)($(cu.south east)+(0.5em,-0.5em)$)},
%          label=above:{\bf Blockchain}] () {};
%    \node[bg-box,
%          tc-server-color,
%          fit={($(enc.north east)+(0.5em,0.5em)$)($(enc.south west)+(-0.5em,-0.5em)$)},
%          label=above:{\bf TC Server}] () {};
%  \end{pgfonlayer}
%\end{tikzpicture}
%\caption{{\bf Gas Payment in Ethereum}}
%\label{fig:gas-model}
%\end{figure}



\begin{figure}
\begin{tabularx}{\linewidth}{|@{\hspace{3pt}}r@{\hspace{1ex}}X@{\hspace{3pt}}|}
  \hline

  \multicolumn{2}{|c|}{{\bf \tcs blockchain contract \tcont with fees}} \\[1ex]
  {\bf Request:} & On recv $({\sf params}, {\sf callback}, \fee, \gasrequest)$ from some $\reqcont$: \\
                 & Assert $\constgasmin \leq \fee \leq \constgasmax$ \\
                 & $\dgid := \text{Counter}$; \ \ $\text{Counter} := \text{Counter} + 1$ \\
                 & Store $(\dgid, \dgform, \dgcallback, \fee)$ \\[-0.8em]
                 & {\it \sgray{//~$\fee$ held by contract}} \\

  {\bf Deliver:} & On recv $(\dgid, \dgform, \dgm, \gasdeliver)$ from $\tcadd$: \\
                 & Retrieve stored $(\dgid, \dgform', \dgcallback, \fee)$ \\
                 & \quad \sgray{\it //~abort if not found} \\
                 & Assert $\dgform = \dgform'$ \\
                 & Assert $\fee \leq \gasdeliver$ \\
                 & Send $\fee$ to \tcadd \\
                 & Set $\gascallback := \fee - \constgasmin$ \\
                 & Call $\dgcallback(\dgm)$ with $\gascallback$ max gas \\
  \hline
\end{tabularx}
\caption{
Town Crier contract \tcont reflecting fees.
The last argument of each entry point is the {\tt GASLIMIT} provided.
An honest requester sets $\fee$ to be the gas amount
required to execute the {\bf Deliver} entry point including $\dgcallback$.
Town Crier sets $\gasdeliver := \constgasmax$, but lowers the gas limit for $\dgcallback$ ensure that no more than $\fee$ is spent.
}
\label{tbl:tc-contract2}
\end{figure}

\begin{figure}[!h]
\begin{boxedminipage}{\columnwidth}
\centering
{\bf Program for \tcs~\encname ($\enclaveprog$)} \\[1ex]
\begin{tabular}{l}
{\bf Initialize}: Same as Figure \ethan{refer} \\[3pt]

{\bf Resume:} On recv $(\resumecall, (\dgid, \dgform))$ \\
\quad Same as before except the last two statements: \\
\quad $\sigma := \Sigma.{\sf Sign}({\skTC}, (\dgid, \dgform, \dgm, \constgasmax))$ \\
\quad Output $((\dgid, \dgform, \dgm, \constgasmax), \sigma)$ \\
\end{tabular}
\end{boxedminipage}
\caption{The \tcs~\encname \engine.}
\label{fig:engineprot}
\end{figure}

\begin{figure}
\centering
\begin{tikzpicture}
  [contract/.style={entity,minimum height=3.5em,text width=7.5em},
   wallet/.style={entity,minimum height=3.5em,text width=5em},
   type-box-color/.style={fill=black!10},
   blocked-out-label/.style={text=black,type-box-color,text height=0.6em}]
  \node[wallet,fill=white] (user) {User ${\cal P}_U$};
  \node[contract,fill=white,right=6em of user] (cu) {User Contract\\$\reqcont$};
  \node[wallet,trusted,below=5em of user] (tc-wallet) {$\tcadd$};
  \node[contract,trusted,right=6em of tc-wallet] (ctc) {TC Contract\\$\tcont$};
  \node[color=maroon,draw,anchor=south east] (fee) at ([xshift=-0.25em,yshift=0.25em]ctc.south east) {\small $\fee$};

  \draw[color=gold,rounded corners,opacity=0.25,line width=1ex] ([yshift=1.25em]user.east) -| ([xshift=3.25em]cu.south);
  \path[-stealth,color=gold,line width=1.15ex] (user) edge [transform canvas={yshift=1.25em}] (cu);
  \path[-stealth,color=gold,line width=1ex] (cu) edge [transform canvas={xshift=3.25em}] (ctc);
  \path[-stealth,color=maroon,line width=0.4ex] (cu.south) edge [left,transform canvas={xshift=3.25em}] node [blocked-out-label,xshift=1em,yshift=1.2em] {\small \smash{$(\gasrequest, \fee)$}} ([yshift=-1.5em]ctc.north);
  \node[anchor=north,xshift=1em,yshift=-0.75em] () at (cu.south) {\small \bf \underline{\smash{Request}}};

  \path[-stealth,color=gold,line width=1ex] (tc-wallet.east) edge [above,transform canvas={yshift=0.4em}] node () {\small $\gasdeliver$} (ctc.west);
  \draw[color=gold,rounded corners,opacity=0.25,line width=0.3ex] ([yshift=0.4em]tc-wallet.east) -| ([xshift=-3em]ctc.north);
  \path[-stealth,color=gold,line width=0.3ex] (ctc.north) edge [right,transform canvas={xshift=-3em}] node [yshift=-1.25em] {\small \smash{$\gascallback$}} (cu.south);
  \path[color=maroon,opacity=0.25,line width=0.4ex] (tc-wallet-|fee.west) edge [transform canvas={yshift=-1.25em}] (tc-wallet.east);
  \path[-stealth,color=maroon,line width=0.4ex] (ctc.west) edge [transform canvas={yshift=-1.25em}] node [blockchain-color,xshift=0.4em] {\small $\fee$} ([xshift=-0.8em]tc-wallet.east);
%  \path[-stealth,color=gold,ultra thick,dashed] (ctc.west) edge [below,transform canvas={yshift=-1em}] node [text=black,xshift=0.4em] {\small $\gasdeliver - \fee$} ([xshift=-0.8em]tc-wallet.east);

  \path[] (tc-wallet.north east) edge [draw=none,above] node [yshift=0.75em] {\small \bf \underline{Deliver}} (ctc.north west);

  \begin{pgfonlayer}{background}
    \node[bg-box,
          blockchain-color,
          fit={($(ctc.south east)+(1em,-1em)$)($(user.north west)+(-1em,2em)$)},
          label=above:{\bf Blockchain}] () {};
    \node[bg-box,
          type-box-color,
          fit={($(ctc.south east)+(0.5em,-0.5em)$)($(cu.north west)+(-0.5em,0.5em)$)},
          label=above:{\bf Contracts}] () {};
    \node[bg-box,
          type-box-color,
          fit={($(tc-wallet.south east)+(0.5em,-0.5em)$)($(user.north west)+(-0.5em,0.5em)$)},
          label=above:{\bf Wallets}] () {};
  \end{pgfonlayer}
\end{tikzpicture}
\caption{{\bf Blockchain Money Flow within Town Crier.}
Red arrows denote flow of money and brown arrows denote gas provided for function execution.
The thickness of the line indicates the quantity of the resource.
The $\gascallback$ is thin because that value is limited by {\bf Deliver} to $\fee - \constgasmin$.
}
\label{fig:money-flow}
\end{figure}



\paragraph{Town Crier protocol with transaction fees.}
Our basic Town Crier system implements a policy where the requester pays for all gas needed and Town Crier effectively pays nothing.
We now describe how this can be realized by modifying the fee-free protocol described in Section \ethan{refer}.

\begin{itemize}[leftmargin=1.5em]
  \item {\it Initialization.}
    We assume that Town Crier deposits at least $\constgasmax$ into the wallet $\tcadd$.

  \item {\it \tcs blockchain contract.}
    Figure \ref{tbl:tc-contract2} describes the \tcs blockchain contract reflecting fees.
    Since $\tcadd$ must invoke {\bf Deliver}, \tc will pays the gas cost.
    It sets the {\tt GASLIMIT} $\gasdeliver := \constgasmax$.
    To ensure that the gas spent will not exceed the reimbursement available ($\fee$),
    \tcont sets the {\tt GASLIMIT} $\gascallback$ for the sub-call to $\dgcallback$ to $\fee - \constgasmin$.

  \item {\it Town Crier Relay.}
    The relay behavior does not change with the presence of fees.
    It still monitors the blockchain and whenever the contract \tcont stores a new request $(\dgid, \dgform, \_, \_, \_)$,
    it invokes $\enclaveprog$ with $\resumecall(\dgid, \dgform)$.

  \item {\it Town Crier enclave.}
    We make the following small modification to the fee-free protocol.
    Instead of signing the tuple $(\dgid, \dgform, \dgm)$ at the end of its execution,
    the enclave now signs the tuple $(\dgid, \dgform, \dgm, \gasdeliver)$ where $\gasdeliver = \constgasmax$.

  \item {\it Requester.}
    An honest requester behaves the same was as in Figure \ethan{refer} except for setting a {\tt GASLIMIT} of $\gasrequest$ and sending $\fee$ money with each request.
    It sets $\gasrequest$ to be at least the cost of executing the {\bf Request} entry point
    and $\fee$ to be the cost of executing the {\bf Deliver} entry point (including executing the user-defined $\dgcallback$ function).
\end{itemize}

These changes do not modify the properties on which our authenticity proof in Section \ethan{ref authenticity} relied, so that result still holds.
We will prove in Section \ethan{refer} that this new protocol ensures that \tc cannot lose money when the \medname is honest
and that an honest requester's loss is limited even against a malicious \tc.




\subsection{Gas Neutrality}


\begin{theorem}[Gas neutrality for Town Crier]
Assuming an honest Town Crier relay,
Town Crier's wallet account $\tcadd$ will have at least $\constgasmax$ remaining after each {\bf Deliver} call.
\end{theorem}

\begin{proof}[(sketch)]
Honest relay means $\resumecall(\dgid, \dgform)$ will always be legitimate and $\dgform = \dgform'$ in {\bf Deliver}.
While $\fee$ is specified by a potentially malicious user, \tcont rejects the request unless $\constgasmin \leq \fee \leq \constgasmax$.
This means $\gasdeliver = \constgasmax \geq \fee$.
We also will never respond to the same request twice (these are specified in the enclave protocol), so we will never abort.

We limit the gas provided to $\dgcallback$ to $\fee - \constgasmin$ (which must be positive because of the check in {\bf Request}),
and $\constgasmin$ is set so that it is at least the cost of {\bf Deliver} not including $\dgcallback$.
This means that the total gas used is no greater than $\constgasmin + (\fee - \constgasmin) = \fee$, which is exactly what $\tcadd$ is sent for reimbursement.
We also don't run out of gas because $\fee \leq \constgasmax = \gasdeliver$, so we cannot use more gas than we provide.
\end{proof}


\begin{theorem}[Bounded loss for honest requester]
For any request $(\dgform, \dgcallback, \fee)$ submitted by an honest user $\reqcont$,
the requester will lose no more than $\gasrequest + \constgasmax$ money.
\end{theorem}

\begin{proof}[(sketch)]
In Ethereum, all money must be sent by the address controlling that money.
An honest user will only send $\gasrequest + \fee$ to a given request and no more with $\fee \leq \constgasmax$.
They will not continue to send money to \tcont without receiving a response, so that means that they cannot lose more than $\gasrequest + \constgasmax$
even if \tc is malicious and does not deliver data.
\end{proof}



