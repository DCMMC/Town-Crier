\subsection{Alt. Intro Text}

\elaine{there needs to be threat model somewhere}

\elaine{Ari, you asked me to write some alternative text for intro, here it is.}

We present the design of Town Crier, an authenticated data feed service for smart contracts. 
Town Crier bridges the gap between data 
feed providers (e.g., various websites)
and smart contracts (e.g., those running on modern blockchains
such as Ethereum). Since the deployment
of many interesting smart contracts 
today are hampered by the lack of authenticated data feeds,
we believe that a service like Town Crier will like
fuel the deployment of a rich spectrum  
of smart contract applications.
Town Crier can provide attestations of the following nature:

\begin{itemize}[leftmargin=3mm]
\item[]
\noindent {``datagram $X$ is  
presented by an HTTPS-enabled website $Y$ 
{\it after} a certain time $T$''}
\end{itemize}

\elaine{we really need to explain this ``after'' somewhere. it is catchy and will
draw criticism if we don't explain it carefully. I also don't think obscurity of abstraction
will help solve this problem, if anything, reviewers will be more upset
if we don't explain this upfront.}

The security of such data feed attestations
are backed by the security of trusted hardware --- in particular,
our current implementation relies on the newly released Intel SGX secure processor.
In designing and building Town Crier, we make the following contributions:

\paragraph{Full-fledged implementation, pending open-source and launch.}
We designed and built a full-fledged
implementation of Town Crier, offering formal security guarantees
(at the cryptography protocol level).
The design and implementation of Town Crier is meant not only to be  
an interesting academic exercise, but we also plan
to launch Town Crier 
in a production environment (namely, atop Ethereum)
in the near future. 
Our current implementation already offers complete features, and is
only awaiting  
the launch of Intel SGX's attestation service  
to go alive.
When launched, Town Crier will offer a free service to smart contract users, in the sense
that the user only needs to pay for the extra (small amount of) gas
needed to invoke the Town Crier
service on the blockchain. 

We also plan to open source all of Town Crier's code --- in this way, 
users will be able to audit Town Crier's SGX enclave implementation
to verify its trustworthiness.


\paragraph{Formal security.}
Smart contracts  typically execute in an adversarial environment
where contractual parties are incentivized to 
deviate from the prescribed protocol, to maximize their financial gains.
Legal recourse is often too costly or 
not possible due to the following reasons:
1) one compelling feature of 
smart contracts is that they enable 
micro-contracts or micro-transactions that 
would not make sense in the presence of high legal fees;
and 2) contractual parties can be arbitrary, pseudonymous users on the Internet.

Due to the above reasons, formal security is of vital importance
in the design of Town Crier.
We offer formal modeling and proofs for Town Crier's security.
In particular, Town Crier formally defines and ensures  
the following types of guarantees:

\begin{itemize}[leftmargin=5mm]
\item
{\it Authenticity}: If a data feed is returned
to a requester contract, the data feed is guaranteed
to truly reflect the state of 
the requested website $Y$ 
at some point $t \geq T$, where $T$ is a parameter  specified 
by the requester.
\item
{\it Gas neutrality for Town Crier.} 
Assuming that Town Crier itself is honest, 
then Town Crier does not lose money   
even when blockchain users or requester contracts
can behave arbitrarily maliciously.
\item
{\it Bounded expenditure for the requester.} 
Even when all other users, contracts, and Town Crier itself 
can be malicious, 
if an honest requester  
fails to obtain a valid data feed, its loss is bounded by a tiny 
amount of gas.
If a requester obtains a valid data feed, besides paying for the gas
needed to execute the requester's own program logic, the honest requester's 
extra expenditure  
is bounded by a tiny amount of gas.
\end{itemize}

To obtain the above formal guarantees,
we adopt the formal modeling 
of blockchains proposed by Kosba et al.~\cite{hawk},
as well as a formal abstraction for SGX proposed by Shi et al. \elaine{cite}.
We reveal interesting challenges and technical subtleties
that arise in the process of formal reasoning.
For example, in Section \elaine{refer}, we show how 
a subtle gap between the formal blockchain
of Kosba et al.~\cite{hawk}
and Ethereum's instantiation  
turns out to be important in the design of Town Crier's
gas handling and its formal reasoning.  \elaine{actually write this.}


\paragraph{Show-case applications and pending launch.}







