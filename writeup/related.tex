\section{Related Work}

Several data feeds are deployed today for smart contract systems such as Ethereum. Examples include PriceFeed~\cite{PriceFeed:2016} and Oraclize.it~\cite{Oraclize:2016}. The latter achieves distributed trust by using a second service called TLSnotary~\cite{TLSnotary}, which digitally signs webpages. These services, however, ultimately rely on the reputations of their (small) providers to ensure data authenticity.  To address this problem, systems such as SchellingCoin~\cite{schellingcoin} and Augur~\cite{augur} rely on mechanisms such as prediction markets to decentralize trust. Prediction markets often rely on human input, though, constraining their scope. They have not yet seen widespread use in cryptocurrencies. 

Despite an active developer community, research community exploration of smart contracts has been very limited to date. Work includes off-block contract execution for confidentiality~\cite{hawk}, and, more tangentially, exploration of e.g., randomness sources in~\cite{bonneau2015bitcoin}. The only exploration relating to data feeds of which we're aware takes the form of consuming (criminal) applications in~\cite{}.

SGX is similarly in its infancy.
While a Windows SDK~\cite{sgxsdk} and programming manual~\cite{sgxmanual} have just been released, a number of pre-release papers have already explored SGX, e.g., \cite{Baumann:2015:SAU:2818727.2799647,7163017,7163052,anati2013innovative,McKeen:2013jv,Phegade:2013km}. Researchers have demonstrated applications include enclave execution of legacy (non-SGX) code~\cite{Haven} and use of SGX in a distributed setting for map-reduce computations~\cite{VC3}. Several works have exposed shortcomings of the security model for SGX~\cite{sgxexplained,sgxsok,shihardwaretalk}, including side-channel attacks and other attacks against enclave state. 


