\vspace{-2mm}
\section{Introduction}
\vspace{-2mm}

Smart contracts are computer programs that autonomously facilitate or execute a contract. For decades, they have been envisioned as a means to  render existing contract law more precise, pervasive, and efficiently executable through logical specification of legal agreements. Szabo, who popularized the term ``smart contact'' in a seminal 1997 essay~\cite{szabosmartcontract}, gave as an example a smart contract that enforces car loan payments. If the owner of the car fails to make a timely payment, a smart contract could programmatically revoke physical access and return control of the car to the bank. %The contract could additionally be programmed to do so only at a time that is safe for the the owner, e.g., while the car is parked.

Cryptocurrencies such as Bitcoin~\cite{bitcoin} have provided key technical provisions for decentralized smart contracts: Direct control of money by programs and fair, automated execution of computations through the decentralized consensus mechanisms underlying  blockchains. 
The recently launched Ethereum furthermore supports Turing-complete code and thus fully expressive, self-enforcing smart contracts, a big step toward the vision of researchers and proponents.  

As Szabo's example shows, however, the most compelling applications of smart contracts require not just blockchain code, but access to data about real-world state and events. Similarly, financial contracts and derivatives, key applications for Ethereum~\cite{yellowpaper,whitepaper}, rely on data about financial markets, such as equity prices. %Applications such as insurance policies are only realizable with data about weather, flights, delivery of goods, and so forth. 

\emph{Data feeds} (a.k.a.~``oracles''), contracts on the blockchain that serve data requests by other contracts~\cite{whitepaper,yellowpaper}, are intended to meet this need. A few data feeds exist for Ethereum today, but provide no assurance of trustworthy data beyond the reputation of their operators (who are typically individuals or small entities), even if their data originates with trustworthy sources. Of course, there exist reputable websites that serve data for non-blockchain applications and sometimes use HTTPS, enabling source authentication of served data. Smart contracts, though, lack network access and thus cannot directly access such data. The lack of a substantive ecosystem of trustworthy data feeds thus remains an oft-cited, critical obstacle to the evolution of in Ethereum and smart contracts 
in general~\cite{commblockstream}.

%\vspace{-2mm}
\paragraph{\bf Town Crier.} We introduce a system called \emph{Town Crier} (\tc) that provides an \emph{authenticated data feed} (ADF) for smart contracts. \tc acts as a high-trust bridge between existing HTTPS-enabled data websites and the Ethereum blockchain. It retrieves website data and serves it to relying contracts on the blockchain as concise, contract-consumable pieces of data (e.g., stock quotes) called \emph{datagrams}. \tc makes use of Software Guard Instructions (SGX),  Intel's recently released trusted hardware capability. It executes its core functionality as a trusted piece of code in an SGX \emph{enclave}, which protects against malicious processes and the OS and can \emph{attest} (prove) to a remote client that the client is interacting with a legitimate, SGX-backed instance of the \tc code. 

Through a smart-contract front end, \tcs respond to requests by contracts on the blockchain with attestations of the following form:

\begin{itemize}[leftmargin=3mm]
\item[]
\noindent {``Datagram $X$ specified by parameters $\dgform$ is  
served by an HTTPS-enabled website $Y$ 
during a specified time frame $T$.''}
\end{itemize}

A relying contract can be assured of the correctness of $X$ in such a datagram if it trusts the security of SGX, the (published) \tc code, and the validity of source data in the specified interval of time.

%\vspace{-2mm}
\paragraph{Contributions of \tc.}
We highlight several important contributions in our design of \tc:

\vspace{2mm}
\noindent{\em Fully functional \tc implementation, with pending open source and launch.}
We designed and implemented \tcs as a complete, end-to-end system that offers formal security guarantees
at the cryptographic protocol level. Aiming beyond an advance in academic research, we plan to launch \tcs
as an open-source, production service atop Ethereum in the near future. Our launch of \tc awaits only availability of the Intel Attestation Service (IAS), which is expected to occur soon. In its initial form, \tcs be a free service for smart contract users, requiring users only to defray the (small) gas cost of invoking \tc on the Ethereum blockchain. 

\vspace{2mm}
\noindent{\em Formal security analysis.}
Formal security is vitally important in a data feed, as  smart contracts execute in an adversarial environment
where contractual parties can reap financial gains by subverting smart contracts and or services they rely upon.
Legal recourse is often impractical precisely because smart contracts beneficially enable micro-services without costly legal setup and enforcement, as well as contracts between arbitrary, pseudonymous users. We thus adopt a rigorous, principled approach to the design of \tcs by formally defining and ensuring: 

\vspace{-1mm}
\begin{itemize}[leftmargin=5mm]
\item
  \setlength{\itemsep}{2pt}
  \setlength{\parskip}{0pt}
  \setlength{\parsep}{0pt}
{\it Authenticity}: A datagram $X$ returned
to a requesting contract is guaranteed
to truly reflect the data served by specified website $Y$ in time interval $T$ with the requester's specified parameters $\dgform$.
\item
{\it Gas neutrality (zero \tc loss).} Assuming that $\tc$ executes honestly, it does not lose money (cannot suffer resource depletion)
even when accessed by arbitrarily malicious blockchain contracts and users.
\item
{\it Fair expenditure (bounded requester loss).} 
Even when all other users, contracts, and \tc itself act maliciously, 
an honest requester will never pay more than a tiny amount beyond what is required for valid computation executed for that request---whether or not a datagram is delivered.
\end{itemize}
\vspace{-1mm}

To obtain the above formal guarantees,
we rely on the formal modeling 
of blockchains proposed by Kosba et al.~\cite{hawk} and the formal abstraction for SGX proposed by Shi et al.~\cite{sgxsok}
Our analysis of \tc reveals interesting challenges and technical subtleties, e.g., 
a subtle gap between the formal blockchain model
of Kosba et al.~\cite{hawk}
and Ethereum's instantiation  
that proves important in formal reasoning about \tc's
gas handling. 

\vspace{2mm}
\noindent{\em Robustness to component compromise.}
\tc minimizes the Trusted Computing Base (TCB) of its trusted code in the SGX enclave.
It thus offers a basic security model in which a user need only trust SGX itself and a designated data source (website).
As an additional feature, \tc can hedge against the risk of a single SGX instance
or website compromise by supporting
various modes of majority voting, 
among multiple 
websites offering the same piece of data (e.g., stock price),  
and/or multiple, possibly geographically dispersed SGX instances.

\vspace{2mm}
\noindent{\em Private and custom datagrams.} To meet the potentially complex confidentiality concerns that can arise in the broad array of smart contracts enabled by \tc, \tc's trusted enclave code is instrumented to ingest confidential user data (encrypted under a \tc public key). It can thereby support {\em private} datagram requests, with encrypted parameters, and {\em custom} datagram requests, which securely access the online resources of requesters (e.g., online accounts) using encrypted credentials. 

\vspace{2mm}
Additionally, to the best of our knowledge, ours is the first research paper reporting implementation of a substantive system on a real, SGX-enabled host, as opposed to an emulator (e.g.,~\cite{Baumann:2015:SAU:2818727.2799647,7163017}).

%\vspace{-3mm}
\paragraph{Applications.} Thanks to the above key contributions, we believe that \tc can spur deployment of a rich spectrum of smart contracts that are hard to realize in the existing Ethereum ecosystem. We present three examples that showcase \tc's capabilities and demonstrate its end-to-end use: (1) A financial derivative (cash-settled put option) that consumes stock ticker data; (2) A flight insurance contract that relies on private data requests about flight cancellations; and (3) A contract for sale of virtual goods and online games (via Steam Marketplace) for ether, the Ethereum currency, using custom data requests to access online user accounts. We experimentally measure response times for associated datagram requests ranging from 192-1309 ms, depending on the datagram type. These times are significantly less than an Ethereum block interval, and suggest that a few SGX-enabled hosts can support \tc data feed rates well beyond the global transaction rate of a modern decentralized blockchain.

%\vspace{-2mm}
\paragraph{\em Organization:} We present basic technical background for \tc (Section~\ref{sec:background}), followed by an architectural description (Section~\ref{sec:architecture}), a basic set of protocols (Section~\ref{sec:protocols}), a discussion of implementation details (Section~\ref{sec:impl}), and formal security analysis (Section~\ref{sec:analysis}). We present three example applications (Section~\ref{sec:applications}) and use these applications as the basis for performance evaluations of \tc (Section~\ref{sec:experiments}).  We present related work (Section~\ref{sec:related}) and conclude (Section~\ref{sec:conclude}) with a brief discussion of future work. The paper appendix includes formalism and future directions omitted from the paper body.




\iffalse

\paragraph{Majority voting.}
Town Crier effectively minimizes the 
Trusted Computing Base (TCB) needed to ensure
the authenticity of the data feed.
The remaining TCB include the Intel SGX 
instance employed as well as the source website. 
To hedge against the risk of a single SGX instance
or website compromised, Town Crier further supports 
various modes of majority voting,  
including voting amongst multiple 
websites offering the same piece of data (e.g., stock price),  
as well as voting among multiple SGX instances possibly
hosted in different geographical locations.



By operating through a smart-contract front end, \tc can thus furnish a requested datagram $X$ from a particular trusted source, e.g., \texttt{https://www.Y.com}, such that a relying contract $\cont$ can authenticate that $X$ came from $Y$ assuming only that: (1) $\prog$ correctly retrieves and serves data and (2) The \tc server's hardware has not been tampered with. (Of course, there is also a need to trust Intel, which serves as a root of trust for SGX-enabled hosts in general.) Note that trust in $Y$ itself is orthogonal, and \tc can combine data sources according to any desired policy (e.g., majority voting across sources).

While conceptually simple, \tc presents a number of technical challenges. First, there are implementation challenges in securely interfacing enclave code with the blockchain and with data sources. Ethereum lacks native support for and therefore does not permit efficient on-chain verification of the proprietary digital signatures SGX uses to generate attestations on enclave code. Enclave code cannot directly scrape blockchain data and must instead rely on an untrusted process to retrieve correct blockchain data.\footnote{This problem cannot be solved by the obvious mechanism of digitally signing blockchain data: The blockchain is globally readable, and thus cannot store private keys, and running a blockchain client in the enclave would bloat the TCB in the enclave unacceptably.} Additionally, for enclave code to communicate securely with data sources, it must do so over HTTPS, yet enclave code in SGX does not control the host's network stack, which is instead controlled by the (potentially untrusted) OS. A second challenge is in the proofs of security for \tc. These proofs require harmonized formalism for SGX and smart contracts that existing work does not yet provide. Moreover, Ethereum requires expenditure of a currency-derived resource known as \emph{gas} to power contracts; thus proving the availability of \tc---a critical property for relying contracts---requires conventional analysis of data integrity coupled with a proof of sustainable gas expenditure. Finally, the use of \tc raises confidentiality challenges. Since blockchain state is globally readable, na\"{i}ve use of \tc would expose potentially sensitive datagram requests, e.g., an flight-insurance policy would publicly reveal the flight information of policyholders, and would render \tc vulnerable to traffic analysis on served data. Much of our contribution in the design of \tc involves addressing these challenges. 

Beyond serving publicly visible requests for generic data, \tc supports private datagram requests, in which the request is encrypted and custom datagram requests in which the requested data comes from an access-controlled system, such as an individual user's account. Thus \tc can support a rich variety of smart contracts, as shown by our implementation of three example contracts: (1) A financial derivative (cash-settled put option) that consumes stock ticker data; (2) A flight insurance contract that relies on private data requests about flight cancellations; and (3) A contract for the sale of virtual goods and online games (via Steam Marketplace) for ether, Ethereum currency, using custom data requests.

As \tc has minimal trust assumptions and requires no modification to existing data source servers or Ethereum, we believe it offers a compelling, practical approach to bootstrapping a data feed ecosystem for smart contracts. 
 
In summary, our contributions are as follows:

\vspace{-1mm}
\begin{itemize}
  \setlength{\itemsep}{2pt}
  \setlength{\parskip}{0pt}
  \setlength{\parsep}{0pt}

\item \emph{\tcs system:} We present and describe our implementation of \tc, an authenticated data feed system that enables smart contracts to obtain datagrams safely from a target data source, specifically an existing HTTPS-enabled server. In contrast to currently available systems, \tc's assurances do not require trust in (typically fledgling) service operators, only in the underlying commodity trusted hardware, namely Intel SGX. 

\item \emph{Interfacing SGX with smart contracts:} We present solutions to the technical challenges that arise in combining two new technologies with incompatible APIs, e.g., the lack of Ethereum support for SGX attestations and the impracticality of authenticating blockchain data, and thus datagram requests, to \tc's trusted enclave program. 

\item \emph{Formal proofs of \tc system security:} We present unifying formalism for the trusted hardware and smart contracts embodied in \tc and provide proofs of security, including the integrity and source authenticity of datagrams and service availability, which requires a new form of analysis of sustainable gas consumption. 

\item \emph{\tc applications:} We implement three practical smart contract applications that benefit from \tc: a financial derivative, a flight insurance contract, and a contract for exchanging online game currency and cryptocurrency. These applications showcase the rich possibilities and capabilities of \tc, including support for private and custom datagrams. 

\end{itemize}

We additionally report on {\em end-to-end experiments} of \tc, involving execution of relying contracts on the Ethereum blockchain, thus validating the system's deployability. Once Intel stands up its Intel Attestation Service (IAS), which is expected to happen in the near future, we will launch \tc as a public service. 




\fi







