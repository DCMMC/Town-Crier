\section{Applications: Requesting Contracts}

\paragraph{Financial Derivatives.}  In order to implement a financial derivative as a smart contract, we require information about the corresponding financial instrument (typically a stock price).  As an example, we implemented a cash-settled put option.  The issuer of the option creates a contract for a particular stock, strike price, time period, and fee.  Customers may purchase the option by sending requests to the contract along with the associated fee indicating the number of units of the option they would like to buy.  Until the expiration date, customers may choose to exercise the put option by making another request to the option contract.  The contract then requests that TC retrieve the closing price of the underlying instrument on the day the option was exercised, and pays out to the customer the difference between the strike price and the closing price for each unit of the option purchased.  To ensure the contract always has sufficient funds to pay out, it must control value of at least the strike price times the maximum number of units sold.

\paragraph{Flight insurance.} Flight insurance, which provides a payout to the purchaser in the event that their flight is delayed or canceled, is a particularly interesting application of smart contracts as it requires  the transmission of data that the customer may wish to keep private, such as the flight number and scheduled time of arrival, via the blockchain.  To accomplish this while maintaining data privacy, the customer encrypts the flight number under a public
key whose corresponding private key is stored only in the \encname.  \encname code is then responsible for determining whether or not there should be a payout and passing the result to the requesting contract stripped of sensitive information.

However, the publicly visible outcome of the contract may leak information about which flight a particular customer was on, especially if the customer received a payout for flight cancellation.  An adversary attempting to compromise privacy may then narrow his search to a list of recently canceled flights.  This can be solved by including two encrypted addresses in the request, one owned by the customer and one owned by the insurance provider.  The \encname passes back to the requesting contract the customer-owned address if the flight is canceled, and the provider-owned address otherwise.  The contract then makes the payout to the returned address, and the adversary gains no new information from the payment so long as he cannot distinguish between the two addresses.

\paragraph{Steam Marketplace.} Steam \kyle{reference?} is an online gaming platform that supports thousands of games and maintains its own marketplace, where users can trade, buy, and sell games and other virtual items.  Through the Steam trading API, for which a key is issued to each user, we can construct a contract that implements the sale of games and items for ether.  A user wishing to sell items creates a contract specifying the items to be sold along with a price in ether for each.  A user wishing to buy the items creates a Steam trade offer for the items, and then makes a request to the contract referencing the trade offer.  The contract then asks TC to verify the contents and status of the trade through the Steam API, and then sends the ether to the seller's address if the trade is successful.  If the trade is unsuccessful, the buyer's ether is returned.

In order to view trade offers through the Steam API, the API key of either the sending or receiving user is required and should be part of the datagram request sent to TC.  As a user's API key should not be publicly exposed, the request should be encrypted as in the flight insurance example above, and \encname code is trusted by the users not to store or use the key for anything other than trade confirmation.

There is a clear parallel between the exchange of virtual goods for ether and the exchange of fiat currency for ether.  The contract remains mostly the same; virtual goods are simply replaced with dollars and the Steam API is substituted out for a (preferably read-only) API for a user's bank statements.  In both cases, the \encname must be trusted not to compromise the user's privacy when given access to their account statements.

%Discuss flight insurance as an example: We'd like to conceal the flight number and date. We might also want to conceal payment, so TC might ingest encrypted addresses and mix them internally.

%Micro-loans too? Linkage to Facebook / Keybase.io

