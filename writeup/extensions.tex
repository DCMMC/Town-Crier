\section{Extensions}

\subsection{Handling Transaction Fees}

To mitigate potential Denial-of-Service (DoS) attacks, 
Ethereum employs a fee mechanism, referred to as ``gas'', 
where the submitter of a transaction (that invokes
an entry point in the contract) pays a  
transaction fee
roughly proportional to the execution time of the 
corresponding entry point.

\paragraph{Notations and assumed execution model.}
In Figure \elaine{fill}, we
use the notation $\sbrown{\Delta F}$
to denote transaction fees (i.e., gas), 
where $\Delta$ is a type annotation 
and ${\tt F}$ denotes the numerical amount of the  
gas. Other non-gas, normal currency units 
are denoted as $\smaroon{\$ F}$ where $\$$ is a type annotation,
and ${\tt F}$ denotes the amount of the currency. 
For simplicity, our notational system assumes 
that gas and normal currency adopt
the same currency unit. 

We assume that the blockchain contract adopts the following execution model
for gas which closely resembles Ethereum's execution model:
\begin{itemize}[leftmargin=5mm]
\item
{\it Providing gas.} 
When a transaction is submitted, it invokes an entry point in the contract.
The transaction submitter provides a gas amount to activate the entry point. 
\item
{\it Extra gas.} 
If extra gas remains at the end of the execution (after invoking an entry point),
all extra gas is refunded to the transaction submitter at the end.
\item
{\it Gas exhaustion.} 
Gas exhaustion is dealt with in the following manner.
Consider each entry point of the contract as a function. 
Functions can call other functions.
Each function can specify a gas upper bound not to exceed
the remaining gas of the parent function (and if left unspecified,
the upper bound is implicitly set to all remaining gas of the parent function).
If execution of the function exhausted the per-function gas 
upper bound, the function execution is aborted and 
state reverted to before the function is invoked.
\end{itemize}

\begin{figure}
\begin{tabularx}{\linewidth}{|@{\hspace{3pt}}r@{\hspace{1ex}}X@{\hspace{3pt}}|}
  \hline

  \multicolumn{2}{|c|}{{\bf Town Crier blockchain contract \tcont with fees}} \\ [1ex]
  {\bf Request:} & On recv $({\sf params}, 
{\sf callback},  
{\color{blue} \Time_{\text{timeout}}},
\sbrown{\Delta F_{\text{request}}}$ $+$ 
$\smaroon{\$F_{\text{deliver}}})$ from some $\reqcont$: \\
		& Assert ${\color{blue} \Time_{\text{timeout}}} > {\sf cur\_time}$\\
		& ${\sf id}$ := Counter; Counter := Counter + 1\\
                 & Record $({\sf id}, {\sf params}, {\sf callback}, \smaroon{\$ F_{\text{deliver}}}, {\color{blue} \Time_{\text{timeout}}}, {\reqcont})$
%\elaine{this makes it look like it's sending money}
\\[-10pt]
    & {\it {\color{gray} {//~at most ${{\Delta F_{\textrm{request}}}}$ {gas consumed}}} }\\[-10pt]
    & {\it {\color{gray} {//~all remaining {gas returned to $\reqcont$}}} }\\[-10pt]
    & {\it {\color{gray} {//~$\smaroon{\$ F_{\text{deliver}}}$} held by contract}} \\
  {\bf Deliver:} & On recv $({\sf id}, {\sf params}, {\sf data}, 
$\sbrown{\Delta {\tt F_{\text{deliver}}}}$ )$ from $\tcadd$: \\
                 & Let $({\sf id}, {\sf params'}, {\sf callback}, \smaroon{\$ F'_{\text{deliver}}}, {\color{blue} \Time_{\text{timeout}}}, \_)$ be the first recorded tuple for ${\sf id}$\\
 & Assert ${\color{blue} \Time_{\text{timeout}}} > {\sf cur\_time}$ \\
                 & Assert ${\sf params} = {\sf params}'$\\
                &   Assert $\smaroon{\$ F'_{\text{deliver}}} \leq \sbrown{\Delta F_{\text{deliver}}}$\\
		& Set ${\sf bDelivered}[{\sf id}]$ \\
                 & Send $\smaroon{\$F'_{\text{deliver}}}$ to $\tcadd$ \\
                 & Call ${\sf callback}({\sf data})$ \\[-10pt]
    & {\it {\color{gray} {//~at most ${{\Delta F_{\textrm{deliver}}}}$ {gas consumed}}} }\\[-10pt]
    & {\it {\color{gray} {//~all remaining {gas returned to $\tcadd$}}} }\\

{\bf Cancel:} & On recv $({\sf id}, \sbrown{\Delta F_{\text{cancel}}})$ 
from some $\reqcont$\\
  & Let $({\sf id}, \_, \_, \smaroon{\$ F_{\text{deliver}}}, {\color{blue} \Time_{\text{timeout}}}, \reqcont')$ be the first recorded tuple for ${\sf id}$ \\
   & Assert $\reqcont = \reqcont'$\\
   & Assert ${\sf bDelivered}[{\sf id}]$ not set \\ 
   & Assert ${\sf cur\_time} \geq {\color{blue} \Time_{\text{timeout}}}$\\
  & Send $\smaroon{\$ F_{\text{deliver}}}$ to \reqcont\\
  \hline
\end{tabularx}
\caption{
Town Crier contract \tcont reflecting fees.
$\sbrown{\Delta F_{\text{request}}}$ denotes the gas for executing the {\bf Request} 
entry point. 
$\sbrown{\Delta F_{\text{deliver}}}$ denotes the gas for executing the {\bf Deliver} entry point
that includes the user-defined ${\sf callback}$.
\smaroon{\$F_{\text{deliver}}} denotes 
${\tt F_{\text{deliver}}}$ amount of 
explicit, non-gas currency units.
Essentially, the requester first pays 
$\smaroon{\$ F_{\text{deliver}}}$ currency units which will be used to refund
the $\sbrown{\Delta F_{\text{deliver}}}$ amount of gas
that $\pksgx$ will need to put in to call the {\bf Deliver} entry point.
}
\label{tbl:tc-contract}
\end{figure}

\begin{figure}[!h]
\begin{boxedminipage}{\columnwidth}
\begin{center}
{\bf Program for Town Crier \medname $\relay$}
\end{center}
\begin{tabular}{l}
{\bf Initialize}: Same as Figure~\ref{fig:relayprot}\\
%Send \initcall to $\fsgx[\enclaveprog, \relay]$\\
%On recv $(\pkTC, \sigatt)$ from $\fsgx[\enclaveprog, \relay]$:\\
%\quad Publish $(\pkTC, \sigatt)$\\[5pt]

{\bf  Loop forever}: \\
Whenever \tcont receives 
a request
$({\sf id}, {\sf params}, \_, \_,$ $\sbrown{\Delta F_{\text{request}}}$ $+ \smaroon{\$ F_{\text{deliver}}})$:  \\  %\sgray{{\it //~{\bf msg.}~$m_2$}}\\
\ \quad Assert $\sbrown{\Delta F_{\text{min}}} \leq  
\smaroon{\$ F_{\text{deliver}}}
\leq \sbrown{\Delta F_{\text{max}}} $ \\
\ \quad Send $(\text{\resumecall}, ({\sf id}, {\sf params}, {\tt F_{\text{deliver}}} )$ to $\fsgx[\enclaveprog, \relay]$ \\
\ \quad On recv $(({\sf id}, {\sf params}, {\sf data}), \sigma)$ from $\fsgx[\enclaveprog, \relay]$:\\ 
\ \quad \quad  
{\sf AuthSend} $(({\sf id}, {\sf params}, {\sf data}, \sbrown{\Delta F_{\text{deliver}}}))$ to \tcont as \tcadd \\
\hspace{50mm} \sgray{\it //~{\bf msg.}~$m_3$}
\end{tabular}
\end{boxedminipage}
\caption{The Town Crier \medname \relay (with fees).}
\label{fig:relayprot}
\end{figure}

\begin{figure}[!h]
\begin{boxedminipage}{\columnwidth}
\begin{center}
{\bf Program for \tcs~\encname ($\enclaveprog$)}
\end{center}
\begin{tabular}{l}
%{\bf Inputs}:  ${\sf params}$, \\[5pt]
%{\bf Initialize}:  On recv (\initcall, $T_0)$: \\ %{\it //~called only once upfront}\\
{\bf Initialize}:  
Same as Figure \elaine{refer}
\\[3pt]

%{\bf Attest}:  On recv \attcall: \\ %{\it //~called only once upfront}\\
%\quad $T := \clock()$\\
%\quad Call quoting enclave with supp.~data $(\pkTC, T)$
%\\[5pt]

{\bf Resume:} On recv (\resumecall, $({\sf id}, {\sf params}, {\tt F_{\text{deliver}}}))$\\
\quad Same as before except the last two statements:\\
\quad $\sigma := \Sigma.{\sf Sign}({\skTC}, ({\sf id}, {\sf params}, {\sf data},
\sbrown{\Delta F_{\text{deliver}}}))$\\
\quad Output $(({\sf id}, {\sf params}, {\sf data}, \sbrown{\Delta F_{\text{deliver}}}), \sigma)$
\end{tabular}
\end{boxedminipage}
\caption{
The \tcs~\encname \engine.
} 
\label{fig:engineprot}
\end{figure}



\paragraph{Town Crier protocol with transaction fees.}
Our basic Town Crier implements a policy where the requester pays for all gas 
needed and Town Crier in effect pays nothing.
We now describe how this can be realized by modifying
the fee-free protocol described in Section \elaine{refer}. 

\elaine{need to add time to all the description below.}
\begin{itemize}[leftmargin=5mm]
\item
{\it Initialization.}
To initialize the system, we assume that Town Crier 
deposits a fixed amount $\sbrown{\Delta F_{\text{max}}}$ 
into the wallet account $\pksgx$.
\item
{\it Town Crier blockchain contract.}
Figure \elaine{refer} describes the  
Town Crier blockchain contract reflecting fees.
Since Town Crier's account 
$\pksgx$ has to invoke the {\bf Deliver} entry point, it has
to advance 
a gas payment 
$\sbrown{\Delta F_{\text{deliver}}}$.
This amount will be entirely refunded through money deposited in the contract 
by the requester.
\item
{\it Town Crier Relay.}
The Town Crier relay monitors
the blockchain, and whenever
the blockchain contract \tcont
receives a new request $({\sf id}, {\sf params}, {\sf callback}, 
\sbrown{\Delta F_{\text{request}}}+\smaroon{\$ F_{\text{deliver}}})$,
it asserts that 
\[
\sbrown{\Delta F_{\text{min}}}
\leq \smaroon{\$ F_{\text{deliver}}} \leq \sbrown{\Delta F_{\text{max}}}
\]
where $\sbrown{\Delta F_{\text{max}}}$ is the total amount of money
in Town Crier's account $\pksgx$, 
and $\sbrown{\Delta F_{\text{min}}}$
is the cost of executing the {\bf Deliver} entry point 
when the user-defined callback is empty.
The check 
$\smaroon{\$ F_{\text{deliver}}} \leq \sbrown{\Delta F_{\text{max}}}$
ensures that Town Crier's enclave  
has sufficient funds to advance
for the {\bf Deliver} phase.
The check 
$\sbrown{\Delta F_{\text{min}}}
\leq \smaroon{\$ F_{\text{deliver}}}$
ensures that 
the {\bf Deliver} entry point should 
have sufficient gas to execute everything excluding the user-defined
callback -- this guarantees that the statement
where Town Crier gets refunded for the gas is always reached.


Finally, the Town Crier relay passes
the tuple $({\tt resume}, 
({\sf id}, {\sf params}, 
{\tt F_{\text{deliver}}}
))$
as input to the enclave.


\item
{\it Town Crier enclave.}
We make the following small modification to the fee-free protocol
described in Figure \elaine{refer}.
Instead of signing the tuple $({\sf id}, {\sf params}, {\sf data})$
at the end of the enclave's execution, the enclave now signs 
the tuple $({\sf id}, {\sf params}, {\sf data}, 
\sbrown{\Delta F_{\text{deliver}}})$
instead, where signing 
$\sbrown{\Delta F_{\text{deliver}}}$ authorizes a
gas amount of $\sbrown{\Delta F_{\text{deliver}}}$ to be advanced
to the contract (which will be refunded later).
\item
{\it Requester.}
The honest requester would behave the same way as in Figure \elaine{refer},
except for additionally putting 
in $\sbrown{\Delta F_{\text{request}}} + \smaroon{\$ F_{\text{deliver}}}$ 
amount with each request, 
where the honest 
requester would set  
$\sbrown{\Delta F_{\text{request}}}$
to the gas cost of executing the  
{\bf Request} entry point,
and set $\smaroon{\$ F_{\text{deliver}}}$ to be the cost 
of executing the {\bf Deliver} entry point (including
the cost of executing the user-defined ${\sf callback}$ function).

If the honest request $\reqcont$ did not receive
a callback for some ${\sf id}$  
before the specified deadline ${\color{blue} \Time_{\text{timeout}}}$,
it will invoke  
{\bf Cancel}
with (${\sf id}$, $\sbrown{\Delta F_{\text{cancel}}}$).
\end{itemize}

\begin{theorem}[Gas neutrality for Town Crier]
Assuming that the Town Crier relay is honest, 
then Town Crier's wallet account $\pksgx$ 
will have at least $\sbrown{\Delta F_{\text{max}}}$
amount remaining after each {\bf Deliver}  
call finishes execution.
\end{theorem}
\begin{proof}[(sketch).]
Since the relay is honest, every time 
$\pksgx$ invokes the {\bf Deliver}
entry point, the following holds:
1) 
$\smaroon{\$ F_{\text{deliver}}}
= \sbrown{\Delta F_{\text{deliver}}}$;
i.e., the gas $\tcadd$ advances is equal
to the fees the 
requester deposited with the \tcont contract;
2)
the amount gas sent 
$\sbrown{\Delta F_{\text{deliver}}} = 
\smaroon{\$ F_{\text{deliver}}}
\leq 
\sbrown{\Delta F_{\text{max}}} 
$;
in other words, 
$\tcadd$ has sufficient funds  
if it starts out with $\sbrown{\Delta F_{\text{max}}}$
in its wallet;
and
3) 
$\sbrown{\Delta F_{\text{min}}} \leq
\sbrown{\Delta F_{\text{deliver}}} = 
\smaroon{\$ F_{\text{deliver}}}
$; in other words, the statement that refunds
$\tcadd$ 
$\smaroon{\$ F_{\text{deliver}}}$
amount will definitely be invoked. This 
ensures that the full gas amount 
$\sbrown{\Delta F_{\text{deliver}}}$
that $\tcadd$ advanced will be refunded (and anything more left
at the end of the execution will also be refunded);
\end{proof}





\begin{theorem}[Bounded loss for honest requester]
Suppose request $({\sf id}, {\sf params}, {\sf callback}, 
{\color{blue} \Time_{\text{timeout}}}, \_)$ was submitted
by an honest requester $\reqcont$;
then 
$\reqcont$ will  either obtain 
a valid datagram matching requested parameters
${\sf params}$
before the specified deadline  
${\color{blue} \Time_{\text{timeout}}}$;
or 
$\reqcont$ will have lost at most 
$\sbrown{\Delta F_{\text{request}}} + \sbrown{\Delta F_{\text{cancel}}}$
after ${\color{blue} \Time_{\text{timeout}}}$.

\end{theorem}

\begin{proof}[(sketch.)]
An honest requester $\reqcont$  
picks a random, sufficiently long ${\sf id}$ such
that the probability of collision with an existing ${\sf id}$ 
is negligible.
Recall that $\tcont$ ignores all future occurrences of ${\sf id}$, i.e.,
only the first occurrence of ${\sf id}$ is preserved. 
Further,
an honest requester $\reqcont$  
also 
puts in 
$\sbrown{\Delta F_{\text{request}}} + \smaroon{\$ F_{\text{deliver}}}$
amount with each request to $\tcont$,
where $\sbrown{\Delta F_{\text{request}}}$
is equal to the gas cost of executing the
{\bf Request} entry point,
and $\smaroon{\$ F_{\text{deliver}}}$ is the gas cost
of executing the {\bf Deliver} entry point (including
the cost of executing the user-defined ${\sf callback}$ function).

Regardless of what the adversary does,  
if the line ``Set ${\sf bDelivered[{\sf id}]}$''
in \tcont
is reached for some ${\sf id}$ at any point,
then it is guaranteed that the following assertions
hold:
${\color{blue} \Time_{\text{timeout}}} > {\sf cur\_time}$,
${\sf params} = {\sf params}'$,
and $\smaroon{\$ F'_{\text{deliver}}} \leq 
\sbrown{\Delta F_{\text{deliver}}}$.
In this case, \tcont will finish execution and the requester
would have obtained the resulting 
datagram for the correct ${\sf params}$ specified in the request,
and before the requested deadline ${\color{blue} \Time_{\text{timeout}}}$.
Else, if this line 
the line ``Set ${\sf bDelivered[{\sf id}]}$''
in \tcont
is never reached for some ${\sf id}$ before the requested deadline 
${\color{blue} \Time_{\text{timeout}}}$, then 
the honest requester $\reqcont$ can invoke the {\bf Cancel}
entry point to obtain a refund 
of $\smaroon{\$F_{\text{deliver}}}$ before ${\color{blue} \Time_{\text{timeout}}}$ ---
in this case the honest requester would have lost
no more than $\sbrown{\Delta F_{\text{request}}} + \sbrown{\Delta F_{\text{cancel}}}$.
\end{proof}

