\section{Experiments}

\begin{itemize}
\item Total execution time--with PROFILING
\item Clock granularity
\item Gas costs
\end{itemize}

\subsection{Gas costs}

At the writing of this paper, 1 ether is worth approximately \$0.20, and 1 gas costs $5 \cdot 10^{-8}$ ether.
This means that 1 million gas costs approximately \$0.01.
Here we give values for our implementation for the gas variables we defined in Section \ethan{ref gas constant definitions} (all units are in gas).
\begin{center}
  \begin{tabular}{lr}
    \hline
    $\constgasmin$ & \num[group-separator={,}]{35000} \\
    \hline
    $\constgasmax$ & \num[group-separator={,}]{3100000} \\
    \hline
    $\constgascancel$ & \num[group-separator={,}]{24500} \\
    \hline
%    $\gascancel$ & \num[group-separator={,}]{38000} \\
%    \hline
%    $\gasrequest$ & \num[group-separator={,}]{120000} + \num[group-separator={,}]{2700}/32 bytes \\
%    \hline
  \end{tabular}
\end{center}
Note that $\constgasmax$ cannot currently go above this level because the Ethereum system will refuse to accept transactions where {\tt GASLIMIT} is above approximately 3.1 million.

The cost of executing the {\bf Cancel} entry point (i.e. the value an honest user will assign to $\gascancel$) is about \num[group-separator={,}]{38000} gas.

The higher expense is due to the high cost of on-blockchain storage.
The scaling is due to the fact that the caller must pay gas for including more bytes in a transaction.
The cost for executing {\bf Request} is approximately \num[group-separator={,}]{120000} gas plus an additional \num[group-separator={,}]{2700} for every 32 bytes of request parameters.


