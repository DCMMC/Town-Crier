\section{Experiments}
\label{sec:experiments}

\begin{table*}[ht]
\centering
\begin{tabular}{lrr|rr|rr}
\toprule
& \multicolumn{2}{c|}{\sf CashSettledPut} &
  \multicolumn{2}{c|}{\sf FlightIns} &
  \multicolumn{2}{c}{\sf SteamTrade} \\
  & time(ms) & proportion & time(ms) & proportion  & time(ms) & proportion\\
\midrule
Context switch & 0.41    & 0.21\%   & 0.44      & 0.08\%    & 0.89    & 0.07\%\\
Web scraper         & 181.03  & 94.12\%  & 512.32    & 98.02\%   & 1297.73 & 99.11\%\\
Sign                & 10.63   & 5.53\%   & 9.64      & 1.85\%    & 10.45   & 0.80\%\\
Serialization       & 0.28    & 0.14\%   & 0.25      & 0.05\%    & 0.32    & 0.02\%\\
\midrule
Total               & 192.35  & 100.00\%    & 522.65    & 100.00\% & 1309.39 & 100.00\%\\
\bottomrule
\end{tabular}
\caption{Running time of handling a request \elaine{Fan to 1) add std dev;
2)update context switch time to include
time switching out too}}
\label{tab:eval_profiling}
\end{table*}



%\begin{table*}[ht]
%\centering
%\begin{tabular}{lrr|rr|rr}
%\toprule
%& \multicolumn{2}{c|}{\sf CashSettledPut} & 
%  \multicolumn{2}{c|}{\sf FlightIns} &
%  \multicolumn{2}{c}{\sf SteamTrade} \\ 
%\midrule
%& \multicolumn{2}{c|}{\bf Enclave response time}
%& \multicolumn{2}{c|}{\bf Enclave response time}
%& \multicolumn{2}{c}{\bf Enclave response time} \\
%  & time(ms) & \multicolumn{1}{c|}{proportion} & time(ms) & \multicolumn{1}{c|}{proportion} & time(ms) & \multicolumn{1}{c}{proportion}\\
%%\midrule
%Context switch & 0.41    & 0.21\%   & 0.44      & 0.08\%    & 0.89    & 0.07\%\\
%Web scraper         & 181.03  & 94.12\%  & 512.32    & 98.02\%   & 1297.73 & 99.11\%\\
%Sign                & 10.63   & 5.53\%   & 9.64      & 1.85\%    & 10.45   & 0.80\%\\
%Serialization       & 0.28    & 0.14\%   & 0.25      & 0.05\%    & 0.32    & 0.02\%\\
%%\midrule
%{\bf Total}     & {\bf 192.35}  & {100.00}\%    & {\bf 522.65}  & 100.00\% & 
%{\bf 1309.39} & 100.00\%\\
%\bottomrule
%\end{tabular}
%\caption{Running time of handling a request \elaine{Fan to 1) add std dev;
%2)update context switch time to include
%time switching out too}}
%\label{tab:eval_profiling}
%\end{table*}
%

We evaluated the performance of \tc on a Dell Inspiron 13-7359 laptop equipped
with an Intel i7-6500U CPU and 8.00GB memory.  This model is chosen 
because it is one of the few SGX-enabled systems available on the market at the
writing of this paper. Although the hosting platform is just moderately preferment, we
show that on a single host, the current implementation is able to handle 
transactions generated by Ethereum blockchain in real time. 
\elaine{todo: modify
the previous sentence 
after Fan's new experiments are done.}


\subsection{Enclave Response Time}
% justify why a laptop is used 
We measured the enclave response time for handling a request. Here ``enclave response time''
is defined as the difference between 1) the time 
the relay decides to send a request to an SGX enclave;
and 2) the time the relay gets a response back from the SGX enclave. 
For each application, we repeat the  
experiment 5 times, and report the mean and the standard deviation.  
\elaine{change accordingly after Fan puts in the std dev}

%The measurement begins
%at the point when a request is discovered and ends at the point when the
%response is sent to delivery. 
%Propagation latency of the Ethereum network
%is not measured since it is out of the control of \tc. 

Table \ref{tab:eval_profiling} summarizes the total enclave response
time as well as its breakdown. 
The table suggests that for the three applications we implemented
the enclave response time ranges from {\bf 192 ms}
to {\bf 1309 ms}.  \elaine{update numbers after Fan updates table}
The response time 
is clearly dominated by the web scraper time, i.e., the time it takes to 
fetch the requested information from a website.
Among the three applications 
evaluated, {\sf SteamTrade} has the longest web scraper time, 
since in the case of {\sf SteamTrade}, the web scraper
must interact with the website over multiple roundtrips to fetch the desired datagram.

%Web scrapers contribute the most to the
%total execution time. This portion of running time
%is application-specific and mainly depends on the
%amount of fetched data.

%Suggested by Table \ref{tab:eval_profiling}, fetching web pages 
%over HTTPS is the most expensive operation. This is because: 1)
%fetching data triggers
%a lot of context switches 
%and data transferring between the enclave and untrusted application;
%2) networking latency is reflected 

\subsection{Transaction Throughput}
We performed a sequence of experiments running 1 to 20 enclaves 
on a single SGX-enable laptop
machine, and investigated the transaction throughput 
enabled by a single SGX machine. 
Note that SGX allows at most 20 enclaves on the specific machine model we used.
Figure \elaine{refer to Fan's new figure} 
shows that for the three applications evaluated,
a single SGX machine can handle
\elaine{blah to blah} 
tx/sec.

We give several meaningful data points of comparison to show
why a single SGX machine suffices to handle the load of
today's blockchains: 
Ethereum as of today handles 
\elaine{fill in} tx/sec on average. 
Bitcoin today handles
\elaine{fill in} tx/sec, and 
its maximum throughput (when the full block size is utilized)
would be roughly 7 tx/sec.
We know of no measurement study that 
investigates the fundamental 
throughput bound of the Ethereum  peer-to-peer network.
However, for Bitcoin, it has been shown that without
redesigning the protocol, 
Bitcoin cannot scale beyond  
42 tx/sec by simple reparametrization of its block size.
\elaine{cite}
Therefore, Town Crier will not be a throughput 
bottleneck when used 
with decentralized blockchains --- it appears 
more imminent to scale up 
the blockchain protocols themselves.

\subsection{Gas costs}
One ether is currently worth about \$5 and 1 gas costs $5 \cdot 10^{-8}$ ether, meaning 4 million gas is approximately \$1.
Here we give values for our implementation for the gas variables we defined in Section \ethan{ref gas constant definitions} (all units are in gas).
In Town Crier, we set $\constgasmin$ to be 35000 gas units (i.e.  \elaine{fill} cents),
and $\constgasmax$ to be 3100000 gas units (i.e., \elaine{fill in} cents --- the latter is limited by 
Ethereum's inherent {\tt GASLIMIT}.

The cost for executing {\bf Request} is approximately \num[group-separator={,}]{120000} 
gas (i.e., \elaine{fill} cents) 
fixed cost, 
plus \num[group-separator={,}]{2500} gas (i.e., \elaine{fill} cents) 
for every 32 bytes of request parameters.


\subsection{Resilience against Component Compromise}

\elaine{add some eval results here.}

\subsection{Clock precision}

Recall that the absolute time of the \encname is verified
by including a signed time stamp in the attestation and letting 
clients check it it in real time. Therefore the latency
of attestation generation plus a round-trip time
define the precision of the absolute clock. 
We evaluated the cycles consumed in generating an attestation and
signing a Unix time stamp.


\begin{table}[ht]
\centering
\begin{tabular}{lr}
\toprule
  & time(ms) \\
\midrule
Report Generation & $18.46$ \\
Quote Generation & $64.94$ \\
Signing the time stamp & $11.50$ \\
\bottomrule
\end{tabular}
\caption{Running time of generating \att}
\label{tab:eval_att}
\end{table}

Table \ref{tab:eval_att} summarizes the running time break down of generating an
attestation along with a signed time stamp.  Note that two steps are involved in
generating \att: first a report is produced within the \encname; then, an
attestation, as known as a quote, is produced by QE based on the report passed
to it. Note that \att only need to be generated once throughout the lifetime of the
enclave.



