\section{Security Analysis}
\label{sec:analysis}

Proofs of theorems in this section appear in Appendix~\ref{sec:analysis-proofs}.


\paragraph{Authenticity.}
Intuitively, authenticity means that an adversary (including a corrupt user, relay, or collusion thereof)
cannot convince \tcont to accept a datagram that differs from the expected content obtained by crawling the specified \weburl at the specified time.
In our formal definition, we assume that the user and \tcont behave honestly.
Recall that the user must verify upfront the attestation $\sigatt$ that vouches for the enclave's public key $\pksgx$.

\begin{definition}[Authenticity of Data Feed]
We say that the \tc protocol satisfies \emph{authenticity of data feed} if,
for any polynomial-time adversary that can interact arbitrarily with $\fsgx$,
it cannot persuade an honst verifier to accept $(\pksgx, \sigatt, \dgform:=(\weburl, \pkurl, T), \dgm, \sigma)$
where $\dgm$ is not the contents of $\weburl$ with the public key $\pkurl$ at time $T$.
More formally, for any probabilistic polynomial-time adversary $\algA$,
\[
\begin{array}{l}
\Pr\left[
\begin{array}{l}
(\pksgx, \sigatt, {\sf id}, {\sf params}, {\sf data}, \sigma) \leftarrow 
\algA^{\fsgx}(1^\lambda) :\\
\quad \left(\sigsgx.{\sf Verify}(\pkM, \sigatt, (\enclaveprog, \pksgx)) = 1\right) \wedge \\
\quad \left(\Sigma.{\sf Verify}(\pksgx, {\sf id}, {\sf params}, {\sf data})  = 1\right) \wedge\\
\quad {\sf data} \neq \enclaveprog({\sf params}) 
\end{array}
\right] \\[3pt] 
\leq {\sf negl}(\lambda)
\end{array}
\]
\label{defn:auth}
\end{definition}


\begin{theorem}[Authenticity]
Assume that $\sigsgx$
and $\Sigma$ are secure signature schemes (recall
that we follow Shi et al. \elaine{cite} who show
how to abstractly  
model SGX's group signature as a regular signature
scheme under a manufacturer public key $\pkM$),
%and assume that the cryptographic protocol used to realize the secure channel
%with $\pkurl$ is secure;
then, the above 
protocol achieves authenticity of data feed by Definition~\ref{defn:auth}.
\end{theorem}




\paragraph{Fee Safety.}
Our protocol in Section~\ref{sec:gas-protocol} ensures that an honest \tcs will not run out of money
and that an honest requester will not pay excessive fees without receiving data.
These properties are encapsulated in the following theorems.

\begin{theorem}[Gas neutrality for Town Crier]
If the \tc~\medname is honest,
Town Crier's wallet $\tcadd$ will have at least $\constgasmax$ remaining after each {\bf Deliver} call.
\end{theorem}


\begin{theorem}[Fair Expenditure for Honest Requester]
For any request $(\dgform, \dgcallback, \fee)$ submitted by an honest user $\reqcont$,
at most $\constgascancel$ of the user's total payment will not be spent on computation requested by and executed on behalf of that user.
\end{theorem}





\subsection{Other security concerns}
We treat side-channel attacks as outside the scope of our initial \tc architecture. Such attacks would be of particular concern should the \medname be compromised. Intel explicitly disclaims protections against side-channel attacks in SGX. The ability for the OS to monitor page faults incurred by a process running in an enclave is an example shown to be potentially serious in practice~\cite{}. Additionally, the \medname or any network adversary can potentially perform traffic analysis to determine what content the \encname is retrieving from a remote server~\cite{}, a potential threat to the confidentiality of private datagrams.


