%\vspace{-3mm}
\section{Background}
\label{sec:background}

In this section, we provide basic background on the main technologies \tc incorporates, namely SGX, TLS / HTTPS, and smart contracts.

%\vspace{-2mm}
\paragraph{\bf SGX.}
Intel's Software Guard Extensions (SGX)~
\cite{McKeen:2013jv,anati2013innovative,Phegade:2013km,sgxmanual}
is a set of new instructions that confer hardware protections on user-level code. SGX enables process execution in a protected address space known as an {\em enclave}.
The enclave protects the confidentiality and integrity of the process from certain forms of hardware attack and other software on the same host, including the operating system. 

A enclave process cannot make system calls, but can read and write memory outside the enclave region. Thus isolated execution in SGX may be viewed in terms of an ideal model in which a process is guaranteed to execute correctly and with perfect confidentiality, but relies on a (potentially malicious) operating system for network and file-system access.\footnote{This model is a simplification: SGX is known to expose some internal enclave state to the OS~\cite{sgxexplained}. Our basic security model for \tc assumes ideal isolated execution, but again, \tc can also be distributed across multiple SGX instances as a hedge against compromise.}

SGX allows a remote system to verify the software in an enclave and communicate securely with it. When an enclave is created, the CPU produces a hash of its initial state known as a {\em measurement}. The software in the enclave may, at a later time, request a report which includes a measurement and supplementary data provided by the process, such as a public key. The report is digitally signed using a hardware-protected key to produce a proof that the measured software is running in an SGX-protected enclave. This proof, known as a {\em quote}, can be verified by a remote system, while the process-provided public key can be used by the remote system to establish a secure channel with the enclave or verify signed data it emits. We use the generic term {\em attestation} to refer to a quote, and denote it by \att. We assume that a trustworthy measurement of the code for the enclave component of \tc is available to any client that wishes to verify an attestation. SGX signs quotes using a \emph{group signature} scheme called 
EPID~\cite{epid}. This choice of primitive is significant in our design of \tcs, as EPID is a proprietary signature scheme not supported in Ethereum.

SGX additionally provides a trusted time source via the function \texttt{sgx\_get\_trusted\_time}.~On invoking this function, an enclave obtains a measure of time relative to a reference point indexed by a nonce. A reference point remains stable, but SGX does not provide a source of absolute or wall-clock time, another limitation we must work around in \tc.

%\vspace{-2mm}
\paragraph{\bf TLS / HTTPS.}
We assume basic familiarity by readers with TLS and HTTPS (HTTP over TLS). As we explain later, \tc exploits an important feature of HTTPS, namely that it can be partitioned into interoperable layers: an HTTP layer interacting with web servers, a TLS layer handling handshakes and secure communication, and a TCP layer providing reliable data stream. 

%\vspace{-2mm}
\paragraph{\bf Smart contracts.}
\label{sec:contracts-and-gas}
While \tc can in principle support any smart-contract system, we focus in this paper on its use in Ethereum, whose model we now explain. For further details, see~\cite{whitepaper,yellowpaper}

A smart contract in Ethereum is represented as what is called a \emph{contract account}, endowed with code, a currency balance, and persistent memory in the form of a key/value store. A contract accepts messages as inputs to any of a number of designated functions. These entry points, determined by the contract creator, represent the API of the contract. Once created, a contract executes autonomously; it persists indefinitely with even its creator unable to modify its code.\footnote{There is one exception: a special opcode \texttt{suicide} wipes code from a contract account.} Contract code executes in response to receipt of a \emph{message} from another contract or a \emph{transaction} from a non-contract (\emph{externally owned}) account, informally what we call a \emph{wallet}. Thus, contract execution is always initiated by a transaction. Informally, a contract only executes when ``poked,'' and poking progresses through a sequence of entry points until no further message passing occurs (or a shortfall in gas occurs, as explained below). The ``poking'' model aside, as a simple abstraction, a smart contract may be viewed as an {\em autonomous agent} on the blockchain.

Ethereum has its own associated cryptocurrency called \emph{ether}. (At the time of writing, 1 ether has a market value of just over \$5 U.S. \cite{ethprice}.)
To prevent Denial-of-Service (DoS) attacks, prevent inadvertent infinite looping within contracts, and generally control network resource expenditure,
Ethereum allows ether-based purchase of a resource called \emph{gas} to power contracts.
Every operation, including sending data, executing computation, and storing data, has a fixed gas cost.
Transactions include a parameter (\texttt{GASLIMIT}) specifying a bound on the amount of gas expended by the computations they initiate.
When a function calls another function, it may optionally specify a lower \texttt{GASLIMIT} for the child call which expends gas from the same pool as the parent.
Should a function fail to complete due to a gas shortfall,
it is aborted and any state changes induced by the partial computation are rolled back to their pre-call state;
previous computations on the call path, though, are retained.

Along with a \texttt{GASLIMIT}, a transaction specifies a \texttt{GASPRICE}, the maximum amount in ether that the transaction is willing to pay per unit of gas. The transaction thus succeeds only if the initiating account has a balance of \texttt{GASLIMIT} $\times$ \texttt{GASPRICE} ether and \texttt{GASPRICE} is high enough to be accepted by the system (miner). 

The management of gas, as we show in our design of \tcs, can be delicate.
Without careful construction of \tc's Ethereum front end, an attacker could exhaust the ether used to power the delivery of datagrams.

Finally, we note that transactions in Ethereum are digitally signed for a wallet using ECDSA on the curve Secp256k1 and the hash function SHA3-256.



